\documentclass{article}
\usepackage{math490-final}
%% Matt needed another kludge
%% vvvv DON'T INCLUDE THIS! vvvv
\def\changemargin#1#2{\list{}{\rightmargin#2\leftmargin#1}\item[]}
\let\endchangemargin=\endlist
%% ^^^^ DON'T INCLUDE THIS! ^^^^
\title{Math 490 Style Guide}
\author{Matthew Kousoulas}
\begin{document}
\maketitle

% bullet installations
% reword latex numbering

This document outlines some basic guidelines for writing \texttt{.tex} files so
that the whole class maintains a uniform look and Amarit is able to compile the
whole answer key at the end of the course.

\section*{Installation}
Included with this guide is a file, \texttt{math490.sty}, that contains code
meant to replace the preamble of your \texttt{.tex} file (explained below).
There are two options for installation:
\begin{itemize}
	\item You can place \texttt{math490.sty} in the same directory as your
		\texttt{.tex} files. This is probably the easier option to get started,
		but if you want to keep problems in multiple directories, you will need
		to copy \texttt{math490.sty} into each directory and possibly update
		them over the course of the semester.

	\item You can also place \texttt{math490.sty} in the \texttt{texmf}
		directory associated with your \TeX{} distribution. The advantage of
		this option is only having a single file to worry about that all of your
		projects can access. On Unix based systems the texmf is usually
		\texttt{\textasciitilde/texmf/tex/latex/math490/math490.sty}. For
		Windows it depends on whether you have MiKTeX or TeX Live. Amarit is
		familiar with MiKTeX and can help you with installation.
\end{itemize}

\section*{Preamble}
The preamble of a tex file is everything before the \verb1\begin{document}1
line. It is important that this be uniform across all the final drafts. (The
technical reason is that \LaTeX{} compilation is weird and we're going to use
the \emph{subfiles} package to make it sane. Each file you send in will have its
preamble replaced with boiler plate code require by \emph{subfiles} and inherit
the preamble from a master file Amarit will write.) The first six lines of your
final drafts should be the following:
\begin{verbatim}
\documentclass{article}
\usepackage{math490}
\title{<title>}
\author{<name>}
\begin{document}
\maketitle
\end{verbatim}
Note that it's fine for you to put shortcuts in or otherwise modify your
preamble for drafts. Just let Amarit or Matt know about these changes so we can
release an updated \texttt{.sty} file before you make the final draft. As long
as the copy Amarit receives to place in the final answer key has the above
format everything will be fine.

\section*{Naming Scheme}
One thing that will be exceptionally helpful for Amarit when he is compiling
the answer key is a consistent naming scheme for files. We decided on
\texttt{<last-name>-<problem-number>.tex} all lowercase. If you are doing
multiple exercises as one assignment, keep them in separate files. As an example
we have included our first homeworks: \texttt{kousoulas-1.2.v.tex} and
\texttt{mutharu-1.1.iii.tex}. These are also included as a demonstration of
\texttt{math490.sty} in addition to the chart below.

\section*{Environments}
We've predefined all of the environments that Riehl uses in the text, theorem,
proof, definition, etc. Unfortunately to get the numbering to work there's a bit
of a kludge. \LaTeX{} automatically numbers theorems and other environments as
they appear in the text. In order to provide the appropriate reference to the
theorem in Riehl's text we need to override this behaviour.

If you want to copy a theorem or definition from the text you need to proceed it
with line \verb1\settheorem{<chapter#>}{<section#>}{<theorem#>}1 and follow it
with \verb1\popthm1. Use this command regardless of whether you are making a
lemma, corollary, definition, or any other environment. The two exceptions are
exercises which are explained below and proofs which are not numbered.

\section*{Exercises}
For readability it would be good for us to preface all of our solutions with a
copy of the exercise statement. Please try to the copy the problem statement
verbatim. Similar to the other environment preface it with
\verb1\setexercise{<chapter#>}{<section#>}{<theorem#>}1, however there is no need
to follow the environment with anything.

\section*{Template}
We've included a template file that you can use to start all your projects. It
contains a lemma, exercise, and proof environment. You only need to change the
body text and numbering. You can also change lemma to a different environment to
match the text.

\section*{Style Conventions}
In the interest of having a professional looking answer key at the end of the
course, we have a handful of formatting conventions largely drawn from Riehl's
text.

First, there are often a variety of notations to choose from for a single
concept. You should defer to Riehl's notation first, and if using something
beyond the text Dr. Pardue's notation second. For example, both \(1_x\) and
\(\mrm{id}_x\) are used to denote the identity morphism on a space \(x\). In
keeping with Riehl's notation please use \(1_x\).

Also, a few things to be careful with in general typesetting. Often \LaTeX{}
will have a bunch of commands that all render the same glyph but have different
spacing. These correspond to whether the glyph is being used to symbolize an
operator, a relation, delimiters, or punctuation. For example \verb1|1 is an
operator while \verb1\mid1 is a relation. Similarly, \verb1:1 is an operator
while \verb1\colon1 is punctuation and is used for function definitions. In the
latter case we have defined a handful of macros for defining functions that
automatically use the correct spacing. This is further documented in the
\LaTeX{} symbol guide that we sent out.

Finally, we want to mention that \verb1$$1 as the delimiter for display style
math mode is deprecated. It is a primitive in \TeX{} that was replaced by
\verb1\[1 and \verb1\]1 by the \textbf{amsmath} packages. (These are
aliases for \verb1\begin{equation*}1 and \verb1\end{equation*}1) The latter
creates a much more robust environment that properly handles spacing and won't
break on complicated commands. \textbf{amsmath} also introduced the matching
syntax \verb1\(1 and \verb1\)1 for in-line math, unlike the above these are just
aliases for \verb1$1, so the choice is stylistic.

\begin{tabular}{l|L}
	Command & Output \\ \hline
	\verb1x | y1 & x|y \\
	\verb1x \mid y1 & x\mid y \\
	\verb1x : y1 & x:y \\
	\verb1x \colon y1 & x\colon y \\
\end{tabular}

\section*{Commands}
We've also defined a bunch of formatting commands to make typesetting easier.
These build on the shortcuts that Dr. Pardue has provided with his own \LaTeX{}
examples. Included below is a table of commands with examples of their use.
We've done our best to include everything that people regularly use. If there is
a shortcut that we've missed and you would like to use, let one of us know and
we will update the \texttt{.sty} file and redistribute it as necessary.
\newpage

\section*{Examples}
To help use these we have a table demonstrating the use of some of the commands.
We've only put in a representative sample. There are a lot more in the
\texttt{.sty} available to use. Please look it over and feel free to make
suggestions. Also note that with the exception of the category names, these
shortcuts are only available in math mode.

\begin{changemargin}{-1in}{-1in}
	\begin{center}
		\begin{tabular}{l|L|l}
			Command & Output & Notes \\ \hline
			\verb1\Top1 & \Top & works in text mode and math mode\\
			\verb1\Setp1 & \Setp & works only in math mode\\
			\verb1\Mod{R}1 & \Mod{R} & the subscript is the issue\\
			\verb1\func{f}{A}{B}1 & \func{f}{A}{B} & functions\\
			\verb1\func{\inv{f}}{PB}{PA}1 & \func{\inv{f}}{PB}{PA} & inverse functions\\
			\verb1\mono{g}{A}{B}1 & \mono{g}{A_{/\ker{f}}}{B} & monomorphisms \\
			\verb1\epic{h}{A}{B}1 & \epic{h}{A}{\range(A)} & epimorphisms \\
			\verb1\Func{F}{\sC^\op}{\cat{D}}1 & \Func{F}{\sC^\op}{\cat{D}} & Riehl uses doubled arrows for functors\\
			\verb4\incl{\io}{S^1}{\RR^2}4 & \incl{\io}{S^1}{\RR^2} & inclusions \\
			\verb1\idfunc{\NN}1 & \idfunc{\NN} & identities \\
			\verb1\oper{+}{\RR}1 & \oper{+}{\RR} & operations \\
			\verb1\nper{\norm{\cdot}}{\CC}{n}1 & \nper{\norm{\cdot}}{\CC}{n} & n-ary operations \\
			\verb1\de\defeq\recip{\ep}1 & \de\defeq\recip{\ep} \\
			\verb1\ZZ\fps{x}1 & \ZZ\fps{x} & automatically inserts \verb1\left1 \& \verb1\right1 \\
			\verb12\isinvb31 & 2\isinvb3 & everyone's favourite relation \\
			\verb1\qty{\frac{\len f}{\rtwo}}1 & \qty{\frac{\len f}{\rtwo}} \\
			\verb1\inlnmat{\al&\be\\\gm&\de}1 & \inlnmat{\al&\be\\\gm&\de} & useful for typesetting permutations \\
			\verb1\angl{\lst{x}{n}}1 & \angl{\lst{x}{n}} & optional argument changes the first index \\
			\verb1\fm{y}{\nil}1 & \fm{y}{\nil} & empty list is the best list \\
		\end{tabular}
	\end{center}
\end{changemargin}

\section*{Packages}
One of the limitations of using subfiles to compile the final project is that we
have to all be using the same set of packages for the whole document. We've
tried to include an expansive list of packages from the start to give everyone
flexibility while still avoiding conflicts and packages that don't come with
both MiKTeX and TeX Live. Here is a list of the packages we've included along
with a brief description:
\begin{description}
	\item[mathtools] is a wrapper for \emph{amsmath} that fixes several
		problems and adds some useful functionality.

	\item[xypic] is a simple drawing library suited to typesetting
		commutative diagrams. It is considerably less powerful than
		\emph{PGF/TikZ} which some of you use, but it is more than enough to
		typeset every diagram in Riehl's book.

	\item[enumitem] allows for customising the enumerate, itemize, and
		description environments. Don't go crazy here, the goal is to match the
		look of the text. We've included a common example.

	\item[array] updates the implementation of the array and tabular
		environments making them extensible and more robust.

	\item[amssymb] provides a handful of extra symbols for typesetting math.

	\item[graphicx] provides the \verb1\scalebox1 command and allows embedding
		of images.
\end{description}
If you look at the \texttt{.sty} file you will see we have included a handful of
other packages. You shouldn't need to interact with these, but we've included a
description for the sake of completeness.
\begin{description}
	\item[ntheorem] is a replacement for \emph{amsthm} allowing more
		flexible typesetting. (Matt couldn't figure out how to get \emph{amsthm}
		to give him output that looked like Riehl's textbook.)

	\item[xspace] is a very simple library that makes some of the shortcuts
		simpler to use. It adds a context aware trailing space to text mode
		shortcuts. See the code for details.

	\item[fontenc] is included along with a bunch of other font packages to
		match the look of Riehl's text. As best as Matt can tell the text uses
		Nimbus Roman for the body text. Some of the spacing is off which annoys
		him to no end, but he can learn to cope.

	\item[geometry] handles the page margins.
\end{description}
Documentation for all of these packages can be found on CTAN, the repository for
\TeX{} packages. We've also included documentation in a separate archive.
\end{document}

