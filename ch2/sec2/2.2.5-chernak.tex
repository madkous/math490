\documentclass[../../main]{subfiles}
\begin{document}

\paragraph{}
\begin{exercise}
	By the Yoneda Lemma, natural endomorphisms of the covariant power set
	functor \(P\colon\textsf{Set}^{op} \to \textsf{Set}\) correspond
	bijectively to endomorphisms of its representing object \(\Omega = \{0,
	1\}\). Describe the natural endomorphisms of \(P\) that correspond to
	each of the four elements of \(\hom(\Omega, \Omega)\). Do these
	functions induce natural endomorphisms of the covariant functor?	
\end{exercise}

\begin{proof}
	The Yoneda Lemma gives the isomorphism $
	\hom(\mathsf{C}(-,c), F) \cong Fc $, for any object $ c $ and
	contravariant functor $ F $ into $ \mathsf{Set} $. Setting $ c $ to be
	$ \Omega $ gives us an isomorphism $ \Psi\colon F\Omega \to
	\hom(\mathsf{C}(-,\Omega), F) $; such that for any $ g \in F\Omega $ we
	obtain a natural transformation $ \Psi(g)\colon\mathsf{C}(-,\Omega)
	\Rightarrow F. $ For any set $ Z $ we can also look at a particular leg
	of this natural transformation, $ \Psi(g)_Z\colon\mathsf{C}(Z,\Omega) \to
	FZ; $ which takes each $ f\colon Z \to \Omega $ to $ Ff(g). $ Now let $ F $
	be $ \mathsf{Set}(-, \Omega). $ This means that $ \Psi(g)_Z:
	\mathsf{C}(Z,\Omega) \to  \mathsf{C}(Z,\Omega) $ takes each $ f\colon Z \to
	\Omega $ to $ \mathsf{C}(f)(g) = f^*g = gf. $

	Recall that $ f $ and $ gf $ both correspond bijectively to elements of
	$ PZ $ under the bijection $ \phi $ defined in Proposition 2.1.6(i)
	that takes functions to the preimage of $ 1 \in \Omega $ under those
	functions. So the endomorphism $ \Psi(g)_Z $ corresponds to the
	endomorphism $ h_Z\colon PZ \to PZ $ that takes $ \phi f $ to $ \phi (gf), $
	for all $ f\colon Z \to \Omega. $ That is to say, $ h_Z $ takes the set $
	\{x \in Z | f(x) = 1\} $ to the set $ \{x \in Z | gf(x) = 1\}. $
	Generalizing for all sets $ Z, $ we have a bijection that takes $
	\Psi(g)\colon\mathsf{C}(-,\Omega) \Rightarrow \mathsf{C}(-,\Omega) $ to $
	h\colon P \Rightarrow P $, such that $ h $ takes $ \{x \in Z | f(x) = 1\} $
	to $ \{x \in Z | gf(x) = 1\}, $ for \textit{all} sets $ Z. $ 

	\begin{itemize}
		\item Let $ h $ be the identity endomorphism $ 1_P $. For any
			set $ Z $ and function $ f\colon Z \to \Omega, $ $ \phi(gf)
			= 1_P\phi(f) = 1_P\{x \in Z | f(x) = 1\} = \{x \in Z |
			f(x) = 1\}. $ So $ gf(x) = f(x) $ for all functions $
			f\colon Z \to \Omega $ and $ x \in Z, $ and this can only be
			the case if $ g = 1_\Omega. $
		\item Let $ h $ be the complement endomorphism $ \mathrm{Comp},
			$ which sends each $ A\in PZ $ to $ Z\backslash A $.
			For any set $ Z $ and function $ f\colon Z \to \Omega, $ $
			\phi(gf) = \mathrm{Comp}\phi(f) = \mathrm{Comp}\{x \in
			Z | f(x) = 1\} = \{x \in Z | f(x) \neq 1\}. $ So $
			gf(x) \neq f(x) $ for all functions $ f\colon Z \to \Omega $
			and $ x \in Z, $ which means $ g $ must be the
			transposition morphism $ \tau\colon\Omega \to \Omega; $
			which sends $ 1 $ to $ 0 $ and $ 0 $ to $ 1 $.
		\item Let $ h $ be the constant endomorphism $ c_\emptyset $
			that sends all $ A\in PZ $ to $ emptyset. $ For any set
			$ Z $ and function $ f\colon Z \to \Omega, $ $ \phi(gf) =
			c_\emptyset(f) = c_\emptyset\{x \in Z | f(x) = 1\} =
			\emptyset. $ So $ gf(x) = 0 $ for all functions $ f\colon Z
			\to \Omega $ and $ x \in Z, $  and this can only be the
			case if $ g = c_0\colon\Omega \to \Omega; $ the constant
			morphism that sends both $ 0 $ and $ 1 $ to $ 0. $ 
		\item Let $ h $ be the constant endomorphism $ c_P $ that sends
			all $ A\in PZ $ to $ Z. $ For any set $ Z $ and
			function $ f\colon Z \to \Omega, $ $ \phi(gf) = c_P\phi(f) =
			c_P\{x \in Z | f(x) = 1\} = Z. $ So $ gf(x) = 1 $ for
			all functions $ f\colon Z \to \Omega $ and $ x \in Z, $  and
			this can only be the case if $ g = c_1\colon\Omega \to
			\Omega; $ the constant morphism that sends both $ 0 $
			and $ 1 $ to $ 1. $
	\end{itemize}

	So $ 1_P $ corresponds to $ 1_\Omega $, $ \mathrm{Comp} $ corresponds
	to $ \tau $, $ c_\emptyset $ corresponds to $ c_0, $ and $ c_P $
	corresponds to $ c_1 $.

	% These functions do not induce endomorphisms of the covariant power set
	% functor because $ \mathsf{C}(\Omega, -) $ does not represent that

	Of the four different functions $h$ above, the only two that are natural
	endomorphisms of the \emph{covariant} power set functor $P$ are the first
	and third.

	The identity is obviously a natural endomorphism. But, so is the constant
	endomorphism $c_\emptyset$. Indeed in this case for any function
	$f\colon e X\to Y$ and any $A\in P(X)$,
	\[P(f)h_X(A)=P(f)(\emptyset)=\emptyset = h_YP(f)(A).\]

	But, $h$ is not a natural endomorphism in the second and fourth cases. For
	example, let $X$ be any nonempty set and $f\colon\emptyset\to X$ be the
	unique such function. If $h=\text{Comp}$, then
	$P(f)h_\emptyset(\emptyset)=P(f)(\emptyset)=\emptyset$ while
	$h_XP(f)(\emptyset)=h_X(\emptyset)=X$. So, $P(f)h_\emptyset\ne h_XP(f)$ and
	$h$ is \emph{not} a natural endomorphism. The very same computation shows
	that if $h$ is the constant function sending every element of $P(Z)$ to $Z$,
	then for $f\colon\emptyset\to X$, $P(f)h_\emptyset\ne h_XP(f)$ and $h$ is
	not a natural endomorphism.
\end{proof}
\end{document}
