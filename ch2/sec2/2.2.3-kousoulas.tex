\documentclass[main.tex]{subfiles}
\begin{document}
\maketitle

\paragraph{}
\begin{exercise}
	As discussed in Section 2.2, diagrams of shape \(\com\) are determined
	by a countably infinite family of objects and a countable infinite sequence
	of morphisms. Describe the Yoneda embedding
	\(\incl{y}{\com}{\Set^{\com^\op}}\) in this manner (as a family of
	\(\com^\op\)-indexed functors and natural transformations). Prove
	directly, without appealing to the Yoneda lemma, that \(y\) is full and
	faithful.
\end{exercise}

\begin{proof}
	It is reasonable to identify the category \(\com\) with the ordinal \(\om\).
	Hence the objects of \(\com\) are the elements of \(\om\) and the morphisms
	of \(\com\) are ordered pairs of objects where the second element is greater
	than or equal to the first. For example, we have elements \(n\) and \(m\)
	with a morphism \(n\le m\).

	Now we describe the action of \(y\) on objects of \(\com\). In this light,
	for a given \(n\) in \(\om\), the functor \(\com(-,n)\) takes an object
	\(m\) and maps it to \(\set{m\le n}\) if this is a true statement, and the
	empty set otherwise. Similarly, given an arrow \(p\le q\), \(\com(-,n)\)
	re-imagines this as a function that pre-composes with \(q\le n\) to give
	\(p\le n\). This is just the unique function between the singleton sets
	\(\set{q\le n}\) and \(\set{p\le n}\). Note that if \(p\) is greater than
	\(n\) (along with \(q\) by transitivity), then both \(p\) and \(q\) have
	been mapped to the empty set and rather than an arrows \(p\le q\), \(q\le
	n\), and \(p\le n\) we have the empty map, which all compose as we want
	them. If only \(q\) is greater than \(n\), then \(p\le n\) becomes a
	nontrivial map under our functor while \(q\le n\) is still the empty map.
	This is not a problem because the empty map is absorbing with respect to
	pre-composition.
	\[\begin{tikzcd}
			0\le n & 1\le n \ar[l] & 2\le n \ar[l] & \cdots \ar[l] &
			n\le n \ar[l] & \nil \ar[l] & \cdots \ar[l]
	\end{tikzcd}\]

	Similarly, \(y\) takes a morphism \(m\le n\) to a map from \(\com(-,m)\) to
	\(\com(-,n)\)
	defined as follows. As shown above, an object of \(\com(-,m)\) is
	\(\set{p\le m}\) or the empty set. This time we post-compose \(p\le m\) with
	\(m\le n\) to get \(p\le n\) (or the empty set analogously to the above
	argument). Concretely \(m\le n\) is a collection of maps of the form
	\(\set{(p\le m),(p\le n)}\) where either the second or both symbols may be
	the empty set.
	\[\begin{tikzcd}
			0\le n & 1\le n \ar[l] & \cdots \ar[l] & m \le n \ar[l] &
			\cdots \ar [l] & n\le n \ar[l] & \nil \ar[l] & \cdots \ar[l] \\
			0\le m \ar[u, "(m\le n)(0\le m)"] &
			1\le m \ar[l] \ar[u, "(m\le n)(1\le m)"] & \cdots \ar[l] &
			m\le m \ar[l] \ar[u, "(m\le n)(m\le m)"] & \cdots \ar[l] &
			\nil \ar[l] \ar[u, "(m\le n)\nil"] &
			\cdots \ar[l]
	\end{tikzcd}\]

	Note that this functor is immediately faithful because our domain category
	is a preorder. To see that it is full, note that every object in \Set that
	objects of \(\com\) map on to are either singleton sets or the empty set.
	Cardinal arithmetic gives us three cases to consider \(1^1\), \(1^0\), and
	\(0^0\). In no case is it possible for there to be a map in the homset that
	our functor missed.
\end{proof}
\end{document}
