\documentclass[main.tex]{subfiles}
\begin{document}

\begin{exercise}
	Use the Yoneda lemma to explain the connection between homeomorphisms of the
	standard unit interval $I = [0,1] \subset R$ and natural automorphisms of the
	path functor $\bold {Path} \colon\bold {Top} \rightarrow \bold {Set}$.
\end{exercise}

\begin{proof}
	Viewing the path functor, $F = \bold {Path}\colon\bold{Top} \rightarrow
	\Set$, in light of the Yoneda lemma, there is a bijection $Hom(
	\bold{Top} (I,-),F) \cong FI$, such that $(\alpha \colon\bold{Top} (I,-)
	\Rightarrow F) \mapsto \alpha_I (1_I)$, and for $x \in FI$ with $y \in ob
	\Top$, $x \mapsto (\psi (x) \colon\Top(I,-) \Rightarrow F)$, with
	components of the natural transformation $\psi(x)_y \colon\bold{Top} (I,y)
	\rightarrow Fy$.

	Now, for paths in the unit interval, with continuous functions $f$ being mapped
	to $Ff(x)$, from the Yoneda lemma again, there is a bijection $Hom(\bold{Path},
	\bold{Path}) \cong \bold{Path} (I)$, with $I \mapsto \Top(I,-) =
	\bold{Path} (I)$. The covariant functor $\Top^{op} \rightarrow
	\bold{Set}^{\Top}$ then gives an isomorphism between endomorphisms of the
	unit interval and natural transformations of the path funtor.

	The Yoneda embedding theorem shows that this is actually a monoid isomorphism
	between $\Hom(\Path,\Path)$ and $\Path(I)=\Top(I,I)$. This gives an isomorphism
	between their groups of invertible elements. In the first case, these are the
	natural automorphisms of $\Path$. In the second, these are the homeomorphisms of
	$I$ to itself.

	% Thus by the
	% Yoneda embedding theorem, endomorphisms on $I$, viewed as mophisms between
	% objects in $End(I)^{\op}$ correspond to natural transformations of the path
	% funtor.

	% Since these endomorphisms can be viewed as monoids, then the isomorphism is
	% also between monoids, and those natural endomorphisms that are natural
	% automorphisms of the path functor, are isomorphic to the homeomorphisms of
	% the unit interval.
\end{proof}

\end{document}
