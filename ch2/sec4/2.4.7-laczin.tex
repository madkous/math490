\documentclass[../../main]{subfiles}

\begin{document}

\paragraph{}
\begin{exercise}
	Prove that the construction of the category of elements defines the action on objects of a functor
	$$
	\func{\int(-)}{\Set^{\cat{C}}}{\cat{CAT}/\cat{C}}.
	$$
	Conclude that if $\func{F,G}{\cat{C}}{\Set}$ are naturally isomorphic, then $\int F \cong \int G$ over $\cat{C}$.
\end{exercise}
\begin{proof}
	$\Set^\cat{C}$ is the category where objects are functors from
	$\cat{C}$ to $\Set$. The category $\cat{CAT}/\cat{C} =
	\int\cat{CAT}(-,\cat{C})$ has as objects functors with codomain
	$\cat{C}$. So, the functor $\int(-)$ takes a functor from
	$\cat{C}\rightarrow\Set$ to a functor with codomain $\cat{C}$.
	
	The construction of the category of elements of a functor
	$\func{F}{\cat{C}}{\Set}$ creates pairs $(c,x)$ where $c\in\cat{C}$ and
	$x\in Fc$. Letting $\int(-)$ take objects to objects by
	$Fc\mapsto(c,Fc)$, we can see that the pair can be viewed as part of
	the data of a functor from $\cat{CAT}\rightarrow\cat{C}$. So, $\int(-)$
	acts on objects by taking the functor $F$ to its data viewed as pairs.
	
	If $\func{F,G}{\cat{C}}{\Set}$ are naturally isomorphic, then there is
	an isomorphism between $Fc$ and $Gc$ for all $c\in\cat{C}$, and in
	their respective categories of elements, we would have that
	$(c,Fc)\cong(c,Gc)$ by the aforementioned isomorphism on the second
	component. Since these pairs are  the contents of $\int F$ and $\int G$
	over $\cat{C}$, we have that $\int F\cong \int G$ over $\cat{C}$.
	
\end{proof}

\end{document}
