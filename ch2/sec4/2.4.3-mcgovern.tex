\documentclass[main.tex]{subfiles}
\begin{document}

\paragraph{}
\begin{exercise}
Use the principle of duality to convert the proof that a covariant functor is
representable if and only if its category of elements has an initial object
into a proof that a contravariant functor is representable if and only if its
category of elements has a terminal object.
\end{exercise}
\begin{proof}
Suppose that $(c,x)$ is a terminal object in $\int F$, where $F\colon \sC^{op}
\rightarrow \Set$. We must show that the natural transformation $\Psi\colon \sC(-,c)
\Rightarrow F$ defined by the Yoneda Lemma is an isomorphism. First, consider
an element $y \in Fd$. We know that we have a unique morphism $f\colon (d,y)
\rightarrow (c,x)$ and so we have a unique morphism from $f\colon d \rightarrow c$
so that $Ff(x) = y$.  Because of our definition of $\Psi(x)$, this means
exactly that $\Psi(x)_d\colon \sC(d,c) \rightarrow Fd$ is an isomorphism. This means
that $\Psi(x)$ is a natural isomorphism and represents $F$.

In the other direction, we want to show that for a natural transformation
$\alpha\colon \sC(-,c) \Rightarrow F$, the pair $(c,\alpha_c(1_c))$ defined by the
Yoneda bijection is a terminal object in the category of elements of $F$.
Because $\alpha_d$ is a bijection, we have for every $y \in Fd$, a unique
morphism $f\colon d \rightarrow c$ where $Ff\colon Fc \rightarrow Fd$ is such that
$Ff(\alpha_c(1_c)) = y$. But this defines a unique homorphism to $(c,
\alpha_c(1_c))$ from every $(d,y)$ in the category of elements, and so $(c,
\alpha_c(1_c))$ is terminal.
\end{proof}
\end{document}
