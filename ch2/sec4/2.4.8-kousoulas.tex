\documentclass[../../main]{subfiles}
\begin{document}
\maketitle

\paragraph{}
\begin{exercise}
	Prove that for any \(\func{F}{\sC}{\Set}\), the canonical forgetful functor
	\(\func{\Pi}{\int F}{\sC}\) has the following property: for any morphism
	\(\func{f}{c}{d}\) in the ``base category'' \(\sC\) and any object \((c,x)\)
	in the fiber over \(c\), there is a unique lift of the morphism \(f\) to a
	morphism in \(\int F\) with domain \((c,x)\) that projects along \(\Pi\) to
	\(f\). A functor with this property is called a \textbf{discrete left
	fibration}.
\end{exercise}

\begin{proof}
	Recall that the objects of \(\int F\) are pairs \((c,x)\) such that
	\(x\in Fx\) and morphisms are pointed maps compatible with the pairs. The
	canonical forgetful functor simply takes \((c,x)\) to \(c\) and a morphism
	to itself. If we are given a morphism \(\func{f}{c}{d}\) and an object
	\((c,x)\), then by the nature of functions, there is a unique \(y\in Fd\)
	such that \(Ff(x)=y\). This \(y\) determines a unique object \((d,y)\) in
	\(\int F\) along with a morphism \(\func{f}{(c,x)}{(d,y)}\). It is clear
	that this is the only morphism satisfying \(Ff(x)=y\).
\end{proof}
\end{document}
