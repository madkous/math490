\documentclass[../../main]{subfiles}
\begin{document}

\paragraph{}

\begin{exercise}Use the defining universal property of the tensor product
	to prove that 
	\begin{enumerate}
		\item $ \mbb{k}\otimes_k\cong V $ for any $ \mbb{k} $ vector space $ V; $ and
		\item $ \qty{U\otimes V}\otimes W\cong U\otimes\qty{V\otimes W} $ for any $ \mbb{k} $ vector spaces $ U,V,\AND W. $
	\end{enumerate}
\end{exercise}
\begin{lemma}
	The tensor product of two vector spaces $ V\AND W $ is spanned by rank one 
	tensors.
\end{lemma}
\begin{proof}
	Rank one tensors have the form $ \sum_{i=1}^n v_i\otimes w_i,$ where
	$ v_i\AND w_i $ are basis vectors are $ V\AND W$ 
	respectively.\footnote{Every vector space has a basis thanks to the Axiom 
	of Choice. The sketch of the proof involves taking a chain of linearly 
	independent subsets and looking at the union of that chain. The union is 
	still a linearly independent set, and therefore an upper bound for this
	chain. Since every chain has an upper bound by Zorn's Lemma (which is 
	equivalent to the Axiom of Choice) a maximal linearly independent set 
	exists. Since a basis can be defined a maximal linearly independent subset 
	of a vector space, we have the every vector space has a basis. }
	These rank one tensors are linearly independent so then can be found in a 
	some basis $ B $ of $ V\otimes W. $ Suppose there was an element of $ 
	V\otimes W $ that is not in	the span of the rank one tensors. Call it $ z. $
	Then $ z \cup B $ is still a linearly independent and can be contained in
	some basis $ B'. $ Since a linear maps is completely defined by its action
	on the basis, we could construct two distinct linear maps that we could 
	factor function $ \func{f}{V\times W}{U} $ through. This contradicts
	the uniqueness of the universal property of the tensor product,
	so the rank one tensors span the tensor product space. 

\end{proof}
	
\begin{proof}
We need to show that for a bilinear map $ f $ there exists a unique linear map 
$ \bar{f} $ that makes the following diagram commute.
$$
\xymatrix{ \mbb{k}\times V \ar[r]^{\otimes}\ar[dr]_f&\mbb{k}\otimes V\ar[d]^{\bar{f}}\\&W	
}
$$
Since $ \func{f}{\mbb{k}\times V}{W} $ is bilinear, for any $ \al\IN\mbb{k}\AND 
v\IN V  $ we have $ f(\al,v) =\alpha f(1,v).$ Because $ \bar{f} $ is linear if we choose $ \bar{f}(\al v)=\al 
\bar{f}(v)=\al f(1,v) $, this diagram will commute. This mapping is 
unique since, it holds for all $ \al,$ we can choose $ \al $ to be one. Since 
we have found the unique $ \bar{f} $ that makes the above diagram commutes,
we can apply the universal property of the tensor product to see that $ 
\mbb{k}\otimes_{\mbb{k}} V\cong V $ as desired. 

Now we will show $ \qty{U\otimes V}\otimes W\cong U\otimes\qty{V\otimes W} $ 
for any $ \mbb{k} $ vector spaces $ U,V,\AND W. $ For some vector space $ X $ 
let $ f $ be a trilinear map $ \func{f}{U\times V\times W}{X}.$ Define
$ \func{f_w}{U\times V}{W} $ to be $ f_w(u,v)=f(u,v,w).$ 
Notice that $ f_w $ is 
a bilinear map from $ U\times V $ to $ X, $ by the universal property of the 
tensor product there exists a unique linear map $ \bar{f}_w $ that makes the 
following diagram commute.
$$
\xymatrix{U\times V\ar[r]^{\otimes}\ar[dr]_{f_w}&U\otimes V\ar[d]^{\bar{f}_w}\\&X
}
$$
%Since, $ f $ is trilinear and
%$$
%\xymatrix{U\times V\ar[r]^{\otimes}\ar[dr]_{f_{w_1}}&U\otimes V\ar[d]^{\bar{f}_{\al 
%w_1}}\\&X}	
%\xymatrix{U\times V\ar[r]^{\otimes}\ar[dr]_{f_{w_2}}&U\otimes 
%V\ar[d]^{\bar{f}_{w_2}}\\&X}
%$$
%$ \bar{f}_{\al w_1} $ is linear, we have 
%$$
%\bar{f}_{\al w_1+w_1}(u,v)=f(u,v,\al w_1+w_2)=\al f(u,v,w_1)+f(u,v,w_2)=\al\bar{f}_{w_1}+\bar{f}_{w_2}.
%$$
Define $ \func{f_L}{(U\otimes V)\times W}{X},$ 
to be bilinear map. Thus $f_L(z,w) 
=\bar{f}_{L_w}(z\otimes 
v_i)$ is a linear map and we can apply the universal property again to get the following 
commutative diagram.
$$
\xymatrix{(U\otimes V)\times W\ar[r]^\otimes\ar[dr]_{f_L}&(U\otimes V)\otimes W\ar[d]^{\bar{f}_{L}}\\&X}
$$
In a similar fashion define $ \func{f^u(v,w)}{V\times W}{X}, $ from $ V\times W $ to an arbitrary 
vector space $ X $ where $ f^u(v,w)=g(u,v,w). $ Again, by the universal 
property of the tensor product there exists a unique linear map $ \bar{f}^u $ 
such that the following diagram commutes.
$$
\xymatrix{U\times V\ar[r]^{\otimes}\ar[dr]_{f^u}&U\otimes V\ar[d]^{\bar{f}^u}\\&X
}
$$
Again we can define $ \func{f_R}{U\times(V\otimes W)}{X}, $
where $ f_R(u,\sum_{i=1}^t\al_iv_i\otimes w_i).$ By the universal property of the 
tensor product we have the following commutative diagram.
$$
\xymatrix{U\times (V\otimes W)\ar[r]^{\otimes}\ar[dr]_{f_R}&U\otimes (V\otimes W)\ar[d]^{\bar{f}_{R}}\\&X
}
$$
Define the maps $ w\mapsto \bar{f}_w  $ from $ W $ to $ \Vect{\mbb{k}} 
(U\otimes V,X)$ and $ u\mapsto \bar{g}_u $ from $ U $ to $ \Vect{\mbb{k}} 
(V\otimes W,X).$ Since $ f\AND g $ are trilinear and $\bar{f}_w \AND \bar{g}_u$ are linear we have
$$
\bar{f}_{\al w_1+w_2}(u,v)=f(u,v,\al w_1+w_2)=\al f(u,v,w_1)+f(u,v,w_2)=\al\bar{f}_{w_1}+\bar{f}_{w_2}
$$
and
$$
\bar{g}_{\al u_1+u_2}(v,w)=g(\al u_1+u_2,v,w)=\al 
g(u_1,v,w)+g(u_2,v,w)=\al\bar{g}_{u_1}+\bar{g}_{u_2}.
$$
By construction $ f\AND g $ for 
$
\Func{\zeta_-}{\cat{Trilin}(U,V,W,-)}{\Vect{k}(U\otimes (V\otimes W),-)}\AND
\Func{\eta_-}{\cat{Trilin}(U,V,W,-)}{\Vect{k}((U\otimes V)\otimes W,-)}
$
we have the following commutative diagrams.
$$
\xymatrix{
	\cat{Trilin}(U,V,W;X)\ar[r]^{g-\circ}\ar[d]_{\zeta_X}&
	\cat{Trilin}(U,V,W;Y)\ar[d]^{\zeta_Y}\\
	\Vect{k}(U\otimes (V\otimes W),X) \ar[r]^{g-\circ}
	&\Vect{k}(U\otimes (V\otimes W),Y)
}
$$
$$
\xymatrix{
	\cat{Trilin}(U,V,W;X)\ar[r]^{f-\circ}\ar[d]_{\eta_X}&
	\cat{Trilin}(U,V,W;Y)\ar[d]^{\eta_Y}\\
	\Vect{k}((U\otimes V)\otimes W,X) \ar[r]^{f-\circ}
	&\Vect{k}((U\otimes V)\otimes W,Y)
}
$$
Each leg in both these natural transformations are isomorphisms, so $ f\AND g $ 
represent the same functor. Thus by Proposition 2.3.1, we must have
that $$ (U\otimes V)\otimes W\cong U\otimes(V\otimes W). $$

\end{proof}

\end{document}