\documentclass[../../main]{subfiles}
\begin{document}

\paragraph{}
\begin{exercise}
What are the universal elements for the representations:
\begin{enumerate}
\item defining the category $\two$ in Example 2.1.5(x)?
\item defining the Sierpinski space in Example 2.1.6(ii)?
\item defining the Sierpinski space in Example 2.1.6.(iii)?
\end{enumerate}
\end{exercise}

\begin{quote}
Definition 2.3.3: A \emph{universal property} of an object $c\in\sC$ is 
expressed by a representable functor $F$ together with a \emph{universal 
element} $x\in Fc$ that defines a natural isomorphism $\sC(c,-)\cong F$ or 
$\sC(-,c)\cong F$, as appropriate, via the Yoneda lemma.
\end{quote}

Note that if $\eta$ is the natural isomorphism from $\sC(c,-)$ or $\sC(-,c)$ to 
$F$, then $x=\eta_c(1_c)$.

\begin{enumerate}
\item 
\begin{quote}
Example 2.1.5: The following covariant functors are representable.\\
(x) The functor $\mor\colon \Cat\rightarrow\Set$ that takes a small category to its set of morphisms is represented by the category $\two$: a functor $\two\rightarrow\sC$ is no more and no less than a choice of morphism in $C$. In this sense, the category $\two$ is the \emph{free} or \emph{walking arrow}.
\end{quote}

In this case, let $f$ be the unique nonidentity morphism in $\two$. Then the 
isomorphism $\eta\colon \Cat(\two,-)\Rightarrow\mor$ is given on an object 
$\sD$ of $\Cat$ by $\eta_\sD\colon \Cat(\two,\sD)\rightarrow\mor\sD$ defined by 
$\eta_\sD F=Ff$. Thus, the universal element is $\eta_\two(1_\two)=1_\two f=f$.

\item 
\begin{quote}
Example 2.1.6: The following contravariant functors are representable.\\
(ii) The functor ${\mathcal O}\colon \Top^\op\rightarrow\Set$ that sends a 
space to its set of open subsets is represented by the \emph{Sierpinski space} 
$S$, the topological space with two points, one closed and one open. The 
natural bijection $\Top(X,S)\cong{\mathcal O}(X)$ associates a continuous 
function $X\rightarrow S$ to the preimage of the open point. This bijection is 
natural because a composite function $Y\rightarrow X\rightarrow S$ classifies 
the preimage of the open subset of $X$ under the function $Y\rightarrow X$.
\end{quote}

To provide a little more detail, say that $S=\{0,1\}$ and that the open sets in 
$S$ are $\emptyset,\{0\},$ and $S$. Note that these open sets satisfy the axioms 
of a topology: $\emptyset$ and $S$ are open, the union of any family of open 
sets is open, and the intersection of any two open sets is open. The closed 
sets are the complements of the open sets: $\emptyset, \{1\},S$. By the open 
point we mean $0$, since the singleton set $\{0\}$ is open, and by the closed 
point we mean $1$, since the singleton set $\{1\}$ is closed.

Recall that for two topological space $Y$ and $X$, $f\colon Y\rightarrow X$ is continuous (i.e. a morphism in $\Top$) if for every open set $U$ in $X$, $f^{-1}(U)$ is open in $Y$. 

In particular, for any continuous function $f\colon Y\rightarrow S$, 
$f^{-1}(0)=f^{-1}(\{0\})$ is open in $Y$. Note that $f$ is completely 
determined by $f^{-1}(0)$ for any element of $y\in Y$ that is not in 
$f^{-1}(0)$ we must have that $f(y)=1$, the only element of $S$ other than $0$. 
Conversely, if $V$ is an open subset of $Y$, then we have a continuous function 
$f_V\colon Y\rightarrow S$ given by 
\[f_V(y)=\begin{cases}0 &\text{if }y\in V\\1 &\text{if }y\notin V.\end{cases}\]
This gives a natural bijection between $\Top(Y,S)$ and $\mathcal{O}(Y)$. That 
is, we have a natural isomorphism $\eta\colon \Top(-,S)\Rightarrow\mathcal O$ 
defined by $\eta_Y\colon \Top(Y,S)\Rightarrow{\mathcal O}(Y)$ defined by 
$\eta_Y(f)=f^{-1}(\{0\})$. 

Now, to the question at hand. We have that the natural isomorphism $\eta$ maps 
to $\eta_S(1_S)=1_S^{-1}(\{0\})=\{0\}$, the open point of $S$ or, more 
precisely, the singleton set consisting of the open point of $S$.
\item 
\begin{quote}
(iii) The Sierpinski space also represents the functor ${\mathcal C}\colon 
\Top^\op\rightarrow\Set$ that sends a space to its set of closed subsets. 
Composing the natural isomorphisms ${\mathcal O}\cong\Top{(-,S)}\cong{\mathcal 
C}$ we see that the closed set and open set functors are naturally 
isomorphic. The composite natural isomorphism carries an open subset to its 
complement, which is closed. This recovers the natural isomorphism described in 
Example 1.4.3(v).
\end{quote}
The solution to this problem is essentially the same as the last part. It is 
worth noting that for topological spaces $Y$ and $X$, a function $f\colon 
Y\rightarrow X$ is continuous if and only if for every closed set $A$ in $X$, 
$f^{-1}(A)$ is closed in $Y$. Indeed, $Y$ is the disjoint union of the inverse 
image of $A$ and the inverse image of the complement $A^c$ of $A$, which is 
open. If $f$ is continuous, then $f^{-1}(A^c)$ is open in $Y$ so that 
$f^{-1}(A)$ is closed in $Y$. Conversely, if $f^{-1}(A)$ is closed in $Y$ for 
every closed set $A$ of $X$, then for every open set $U$ of $X$, $f^{-1}(U^c)$ 
is closed so that $f^{-1}(U)$ is open and $f$ is continuous.

Just as above, we make a natural isomorphism from $\Top(-,S)$ to $\mathcal C$ 
by defining $\epsilon_Y\colon \Top(Y,S)\rightarrow{\mathcal C}(Y)$ by 
$\epsilon_Y(f)=f^{-1}(1)=f^{-1}(\{1\})$. Then under the Yoneda correspondence, 
$\epsilon$ maps to $\epsilon_S(1_S)=1_S^{-1}(\{1\})=\{1\}$, the closed point of 
$S$ or, more precisely, the singleton set consisting of the closed point of $S$.
\end{enumerate}

\end{document}
