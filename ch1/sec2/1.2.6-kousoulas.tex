\documentclass[main.tex]{subfiles}
\begin{document}

\paragraph{}
\begin{exercise}
	Prove that a morphism that is both a monomorphism and a split epimorphism is
	necessarily an isomorphism. Argue by duality that a split monomorphism that
	is an epimorphism is also an isomorphism.
\end{exercise}
	\[\xymatrix{x \ar@(ul,dl)[]_{gf} \ar@/^8pt/[rr]|f && y \ar@/^8pt/[ll]|{g} \ar@(dr,ur)_{1_y}}\]
\begin{proof}
	Let \(\sC\) be a category with objects \(x\) and \(y\) and a morphism
	\(\func{f}{x}{y}\). If \(f\) is a split epimorphism, then there exists
	another morphism \(\func{g}{y}{x}\) such that \(fg=\id_y\). If \(f\) is also
	a monomorphism, then for any object \(w\) and any parallel pair of morphisms
	\(\pfunc{h,k}{w}{x}\), \(fh=fk\) implies that \(h=k\). Combining these facts
	with some basic algebra:
	\begin{align*}
		\id_yf&=f\id_x && \textrm{definition of identies}, \\
		fgf&=f\id_x && \textrm{\(g\) is a right inverse of \(f\)}, \\
		gf&=\id_x && \textrm{\(f\) is left cancellable},
	\end{align*}
	gives that \(g\) is also a left inverse of \(f\).

	Suppose instead that \(f\) is an epimorphism and a split monomorphism with
	left inverse \(g\) in the category \(\sC\). Then it is also a monomorphism
	and a split epimorphism in \(\sC^\op\), thus \(f\) is an isomorphism in
	\(\sC^\op\) and an isomorphism in \(\sC\).
\end{proof}
\end{document}

