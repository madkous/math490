\documentclass[../../main]{subfiles}
\begin{document}

\paragraph{}
% commented out to fix toc
% \setexercise{1}{3}{6}
% \begin{exercise}(For Reference.)
% 	Given functors $\func{F}{\cat{D}}{\cat{C}}$ and
% 	$\func{G}{\cat{E}}{\cat{C}}$, show that there is a category called the
% 	\textbf{comma category} $F\downarrow G$, which has\\
% 	\begin{itemize}
% 		\item as objects, triples $(d\in\cat{D}, e\in\cat{E},
% 			\func{f}{Fd}{Ge}\in\cat{C})$, and
%
% 		\item as morphisms $(d,e,f)\rightarrow(d',e',f')$, a pair of morphisms
% 			$(\func{h}{d}{d'},\func{k}{e}{e'})$ so that the following square
% 			commutes in $\cat{C}$, i.e., so that $f' Fh = Gk f$.
% 			\[\xymatrix{ Fd\ar[r]^f\ar[d]_{Fh} & Ge\ar[d]^{Gk}\\
% 					Fd'\ar[r]_{f'} & Ge' }\]
% 			Define a pair of projection functors $\func{\dom}{F\downarrow G}{D}$
% 			and $\func{\cod}{F\downarrow G}{E}$.
% 	\end{itemize}
% \end{exercise}

\setexercise{1}{4}{5}
\begin{exercise}
	Recall the construction of the comma category for any pair of functors
	$\func{F}{\cat{D}}{\cat{C}}$ and $\func{G}{\cat{E}}{\cat{C}}$ described in
	Exercise 1.3.vi. From this data, construct a canonical natural
	transformation $\Func{\alpha}{F\dom}{G\cod}$ between the functors that form
	the boundary of the square
	\[\xymatrix{
			F\downarrow G\ar[r]^{\cod} \ar[d]_{\dom}& E\ar[d]^G\\
			D\ar[r]_{F}\ar@{=>}[ur]^{\alpha}& C
		}\]
\end{exercise}

\begin{proof}
	Letting $c=(d,e,\func{f}{d}{e}\in C)$ as above, and a morphism
	$m=(\func{h}{d}{d'},\func{k}{e}{e'})$, we can describe the actions of the
	functors $F\dom$ and $G\cod$ on objects:
	\begin{itemize}
		\item $F\dom c = Fd$,
		\item $G\cod c = Ge$,
	\end{itemize}
	and their actions on morphisms:
	\begin{itemize}
		\item $F\dom m = Fh$,
		\item $G\cod m = Gk$.
	\end{itemize}
	From the definition of a natural transformation, we need an
	$\func{\alpha}{\ob F\downarrow G}{\mor\cat{C}}$, and we can get this by
	taking $(d,e,f)\mapsto f\in \cat{C}$. Then, $\func{\alpha_c}{Fd}{Ge}$ in
	$\cat{C}$. From this, we can construct the following diagram:
	\[\xymatrix{
		Fd\ar[r]^{\alpha_c}\ar[d]_{Fh} & Ge\ar[d]^{Gk}\\
		Fd'\ar[r]_{\alpha_{c'}} & Ge'
	}\]
	Which is precisely the diagram of the comma category with $f=\alpha_c$ and
	$f'=\alpha_{c'}$, and from the condition that $f'Fh = Gk f$, we
	have that this diagram commutes. Thus, $\alpha$ is a natural
	transformation.
\end{proof}
\end{document}
