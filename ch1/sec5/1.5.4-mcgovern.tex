\documentclass[main.tex]{subfiles}
\begin{document}

\paragraph{}
\begin{exercise}
	Show that a full and faithful functor \(\func{F}{\sC}{\sD}\) both
	\textbf{reflects} and \textbf{creates isomorphisms}. That is, show:
	\begin{enumerate}
		\item If \(f\) is a morphism in \(\sC\) so that \(Ff\) is an isomorphism
			in \(\sD\), then \(f\) is an isomorphism.

		\item If \(x\) and \(y\) are objects in \(\sC\) so that \(Fx\) and
			\(Fy\) are isomorphic in \(\sD\), then \(x\) and \(y\) are
			isomorphic in \(\sC\).
	\end{enumerate}
\end{exercise}
\begin{proof}
	Consider categories $C$ and $D$, and a functor $F: C \rightarrow D$ that is
	full and faithful. That is, for all $x,y \in C$, the function $F: C(x,y)
	\rightarrow D(Fx,Fy)$ that takes $f$ to $Ff$ is bijective. Now, for a
	morphism $f: x\rightarrow y$, suppose that $Ff: Fx \rightarrow Fy$ is an
	isomorphism. This means that there exists $G: Fy \rightarrow Fx$ where
	$G(Ff) = id_{Fx}$ and $Ff(G) = id_{Fy}$. Now, we can apply the definition of
	full and faithful functor to see that $C(y,x)$ is in bijection with
	$D(Fy,Fx)$ and so there exists a unique $g \in C(y,x)$ where $Fg = G$. We
	claim that $g = f^{-1}$.

	To show this, we must show that $fg = id_y$. We consider $F(fg)$, the image
	of $fg$ under our full and faithful functor. We see that
	$$ F(fg) = FfFg = FfG = id_{Fy} = F(id_y)$$
	by properties of functors and our previous defintions of $g$ and $G$. Since
	we know that $F: C(y,y) \rightarrow D(Fy,F,y)$ is bijective, $F(fg) =
	F(id_y)$ implies that $fg = id_y$. We must also show that $gf = id_x$. We
	use a similar method and show that
	$$F(gf) = FgFf = G(Ff) = id_{Fx} = F(id_x).$$
	Since we again know that $F: C(x,x) \rightarrow D(Fx,Fx)$ is bijective, this
	implies that $gf = id_x$. So we have that $fg = id_y$ and that $gf = id_x$.
	Therefore, $f$ is an isomorphism with inverse $g$.

	\subparagraph{}
	If $Fx$ and $Fy$ are isomorphic, we know that there exists
	some isomorphism $G: Fx \rightarrow Fy$. But since $F$ is a full and
	faithful functor and therefore $C(x,y)$ is in bijection with $D(Fx,Fy)$, $G
	= Ff$ for some $f: x \rightarrow y$. By part i, we know that if $Ff$ is an
	isomorphism, then $f$ is also an isomorphism, so we see here that we have an
	isomorphism $f: x \rightarrow y$, and therefore $x$ and $y$ are isomorphic.
\end{proof}
\end{document}
