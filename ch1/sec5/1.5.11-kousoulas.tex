\documentclass[../../main]{subfiles}
\begin{document}
\subparagraph{\(\Ring\to\Group\)}
\begin{proof}
	An isomorphism of groups must preserve cardinality, among other things.
	Since there are groups of any finite order (consider the cyclic groups), to
	disprove that the functor is essential surjective it suffices to show that
	no ring can have a multiplicative group of a specific order. In particular
	we will consider five.

	First, note that in any ring we may consider the multiplicative order of
	\(-1\), the additive inverse of 1. If 1 is distinct from -1, then -1 has
	order 2,\footnote{We have
		\(0=-1\cdot 0=-1(-1+1)=-1\cdot-1+-1\cdot1=-1\cdot-1+-1\) implying that
	\(-1\cdot-1\) is the additive inverse of \(-1\), so \(-1\cdot-1=1\).}
	implying that the multiplicative group of our ring must contain a subgroup
	of order two, and thus must have even order by Lagrange's theorem. We thus
	need only consider rings where 1 does not have a distinct additive inverse,
	i.e. rings of characteristic two.

	Suppose that we have a ring \(R\) of characteristic two. If the
	multiplicative group of \(R\) has order five, then it must be isomorphic to
	\(\ZZ/5\ZZ\) and thus have some element \(\ze\) with multiplicative order
	five, i.e. \(\ze^5-1=0\).

	Now consider the polynomial ring \(\FF_2\adj{x}\), and the evaluation map
	\(\func{ev_\ze}{\FF_2\adj{x}}{R}\) which takes \(x\) to \(z\). The
	polynomial \(f(x)=x^5-1\) must be in the kernel of this map, and \(x^5-1\)
	factors into \(x+1\) and \(x^4+x^3+x^2+x+1\), both irreducible in
	\(\FF_2\adj{x}\).\footnote{To see this note first that it is neither
		divisible by \(x\) nor \(x+1\). So if it were divisible it would be so
		by two degree two polynomial. However, \(x^2+x+1\) is the only
		irreducible degree two polynomial in \(\FF_2\adj{x}\), and
	\((x^2+x+1)^2=x^4+x^2+1\).}

	\(\FF_2\adj{x}\) is a principle ideal domain, so \(x^5+1\) must be contained
	in the ideal generated by itself or one of its factors.

	\dots

	% Since \(f\) is irreducible and of degree
	% four, the quotient \(\FF_2\adj{x}/f\) is the field \(\FF_{16}\).

	So we may factor the evaluation map through the quotient of \(\FF_2\adj{x}\)
	by \(\ker ev_\ze\), which must then embed \(\FF_{16}\) in \(R\) meaning that
	it's multiplicative group has far more than just five elements. Thus the
	group \(\ZZ/5\ZZ\) is completely missed by our functor which fails to be
	essentially surjective.

	\subparagraph{}
	Now let us consider whether this functor is full or faithful. First,
	consider the ring of real numbers \(\RR\) with their usual operations, this
	has no non-identity homomorphisms. However, if we consider only
	\(\RR^\times\), then there are many group homomorphisms. A typical
	homomorphism is raising an element to some power. Thus our functor cannot be
	full.

	\subparagraph{}
	Finally, let \(R\) be a nonzero ring. For any polynomial \(p\) with
	coefficients in \(R\), the is a ring endomorphism of \(R\adj{x}\) which
	takes \(x\) to \(p\) and thus any other polynomial \(q\) to \(q(p)\). Note
	that most of these are note the identity homomorphism. However, if \(q\) is
	a constant polynomial, i.e. just an element of \(R\), then \(q(p)=q\). So
	these endomorphism are all the identity when restricted to the inclusion of
	\(R\) in \(R\adj{x}\).

	Further, the units of \(R\adj{x}\) are precisely the units of
	\(R\).\footnote{Given an invertible element of \(R\), its inverse is
		retained on inclusion in \(R\adj{x}\) making it a unit of \(R\adj{x}\).
		Conversely, multiplying non-constant polynomials increases their degree
		meaning they cannot multiply to 1, so these are all the units of
	\(R\adj{x}\).} This means that any of the endomorphisms above will also be
	the identity on the units of \(R\adj{x}\) which means that our functor is
	not injective on hom-sets.
\end{proof}
\end{document}

