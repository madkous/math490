\documentclass[main.tex]{subfiles}
\begin{document}

\paragraph{}
\setexercise{1}{6}{3}
\begin{exercise}
	Show that any faithful functor reflects monomorphisms. That is, if
	\(\func{F}{\sC}{\sD}\) is faithful, prove that if \(Ff\) is a monomorphism
	in \(\sD\), then \(f\) is a monomorphism in \(\sC\). Argue by duality that
	faithful functors also reflect epimorphisms. Conclude that in any concrete
	category, any morphism that defines an injection of underlying sets is a
	monomorphism and any morphism that defines a surjection of underlying sets
	is an epimorphism.
\end{exercise}
\begin{proof}
	Suppose that \(\func{Ff}{Fx}{Fy}\) is a monomorphism in \(\sD\), i.e. for
	any object \(w\) in \(\sC\) and parallel morphisms \(\pfunc{h,k}{w}{x}\)
	then \(FfFh=FfFk\) implies that \(Fh=Fk\). (Because the property holds over
	all \(\sD\), it holds in particular over the image of \(F\) in \(\sD\).)
	\[
		\xymatrix{w\ar@2[d]_F\ar@<2pt>[r]^h\ar@<-2pt>[r]_k&x\ar[r]^f\ar@2[d]_F&y\ar@2[d]_F\\
		Fw\ar@<2pt>[r]^{Fh}\ar@<-2pt>[r]_{Fk}&Fx\ar[r]_{Ff}&Fy}
	\]
	Now, supposing that \(fh=fk\) in \(\sC\), this implies that \(F(fh)=F(fk)\)
	by elementary properties of equality. Then by the functoriality axioms we
	have \(FfFh=FfFk\), and by the fact that \(Ff\) is a monomorphism \(Fh=Fk\).
	Finally since \(F\) is faithful and thus injective on \(\sC(w,x)\), \(h=k\).
	This argument amounts to pushing equality clockwise around the above
	diagram.

	Note that this also proves that a functor reflects epimorphisms, since an
	epimorphisms is just a monomorphism in the opposite category. The argument
	above will compose neatly with applying the \(\op\) functor at the beginning
	and end to transport us to the right category.

	Further, given a concrete category \(\sC\) we have a faithful functor from
	\(\sC\) to \(\Set\). Since monomorphisms in \(\Set\) are completely
	characterised by injectivity, this becomes sufficient condition for a map to
	be a monomorphism in \(\sC\). Similarly, surjectivity in \(\Set\) will force
	epic in \(\sC\). Frequently, the faithful functor in question is just the
	forgetful functor and the maps in \(\sC\) are just functions defined on sets
	with peculiar features. Thus it is sensible to talk about injectivity and
	surjectivity in \(\sC\) itself, and we can say that injectivity and
	surjectivity are sufficient---but not necessary---conditions for a map to
	monic or epic respectively.
\end{proof}
\end{document}
