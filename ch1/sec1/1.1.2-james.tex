\documentclass[../../main]{subfiles}
\begin{document}

\settheorem{1}{1}{11}
\begin{definition}
	A \tbf{groupoid} is a category in which every morphism is an isomorphism.
\end{definition}
\popthm

\begin{exercise}
	Let \(\sC\) be a category. Show that the collection of isomorphisms in
	\(\sC\) defines a subcategory, the \tbf{maximal groupoid} inside \(\sC\).
\end{exercise}

\begin{proof}
	Let \(\sC_{\iso}\) denote the objects of \(\sC\) together with its
	isomorphisms. We wish to show that \(\sC_{\iso}\) is a category.
	$\sC_{\iso}$ inherits composition and associativity from $\sC$.
	Notice that the identity morphism for each object
	is in $\sC$ is clearly an isomorphism as it is both right and left
	invertible, so the identity morphisms for each object are in $\sC_{\iso}.$
	Because composition in $\sC_{\iso}$ is the same as in $\sC$, each object
	will have the same identity morphism as in $\sC.$
	To show $\sC_{\iso}$ is closed under composition, take two morphisms
	\(\func{f}{x}{y}\) and \(\func{g}{u}{x}\)
	in \(\sC_{\iso}\). Since \(f\) is an isomorphism, then there is a morphism
	\(h\in\mor\sC_{\iso}\) with \(\func{h}{y}{x}\), such that \(fh=1_y\) and
	\(hf=1_x\). Likewise, since \(g\) is an isomorphism, then there is a
	morphism \(j\in\mor\sC_{\iso}\) with \(\func{j}{x}{u}\), such that \(gj=1_x\)
	and \(jg=1_u\). We can take the composition \(fg\), since
	\(\dom(f)=\cod(g)\). We also have the composition \(jh\), since
	\(\dom(j)=\cod(h)\). And again, respecting domains and codomains, we have
	the composition \((fg)(jh)\), since \(\dom(fg)=\cod(jh)\). From the
	associativity of the parent category \(\sC\), then
	\((fg)(jh)=f(gj)h=f(1_x)h=fh=1_y\). Thus \(jh\) is the right inverse of the
	composition \(fg\). Similarly, since \(\cod(fg)=\dom(jh)\), we have the
	composition \((jh)(fg)\), which again from the associativity of the category
	\(\sC\), \((jh)(fg)=j(hf)g=j1_xg=jg=1_u\). So, \(jh\) is the left inverse of
	\(fg\), and \(fg\) is an isomorphism.

	We have shown that \(\sC_{\iso}\) is a category, with all of the objects of
	\(\sC\) and morphisms of \(\sC\) restricted to the isomorphisms of \(\sC\).
	So the groupoid \(\sC_{\iso}\) is a subcategory of \(\sC\). Presented with
	any other subcategory \(\sD\), of \(\sC\), that is strictly larger than
	\(\sC_{\iso}\), there must be a morphism in \(\sD\) that is not in
	\(\sC_{\iso}\). Then this morphism must not be an isomorphism, and hence,
	\(\sD\) cannot be a groupoid. So, the category \(\sC_{\iso}\) is the maximal
	groupoid that is a subcategory of \(\sC.\)
\end{proof}
\end{document}
