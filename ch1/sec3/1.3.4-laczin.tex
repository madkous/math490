\documentclass[../../main]{subfiles}
\begin{document}

\paragraph{}
\settheorem{1}{2}{3}
\begin{lemma}
	The following are equivalent:
	\begin{enumerate}[(i)]
		\item $\func{f}{x}{y}$ is an isomorphism in $\cat{C}$.

		\item For all objects $c\in\cat{C}$, post-composition with $f$ defines a
			bijection \[\func{f_*}{\cat{C}(c,x)}{\cat{C}(c,y)}\]

		\item For all objects $c\in\cat{C}$, pre-composition with $f$ defines a
			bijection \[\func{f^*}{\cat{C}(y,c)}{\cat{C}(x,c)}\]
	\end{enumerate}
\end{lemma}

\settheorem{1}{3}{1}
\begin{definition}
	A \emph{functor} $F$ from $\cat{C}$ to $\cat{D}$ is a functor
	$\func{F}{\cat{C}}{\cat{D}}$. Explicitly, this consists of the following
	data:
	\begin{itemize}
		\item An object $Fc\in \cat{D}$, for each object $c\in \cat{C}$.

		\item A morphism $\func{Ff}{Fc}{Fc'}\in \cat{D}$, for each morphism
			$\func{f}{c}{c'}\in \cat{C}$, so that the domain and codomain of
			$Ff$ are, respectively, equal to $F$ applied to the domain or
			codomain of $f$.
	\end{itemize}
	The assignments are required to satisfy the following two
	\emph{functoriality axioms}:
	\begin{itemize}
		\item For any composable pair $f$, $g$ in $\cat{C}$, $Fg Ff =
			F(g f)$.

		\item For each object $c$ in $\cat{C}$, $F(1_c) = 1_{Fc}$.
	\end{itemize}
\end{definition}
The functors defined in $1.3.1$ are called \emph{covariant} so as to distinguish
them from another variety of functor that we now introduce.

\settheorem{1}{3}{5}
\begin{definition}
	A \emph{contravariant functor} $F$ from $\cat{C}$ to $\cat{D}$ is a functor
	$\func{F}{\cat{C}^{\op}}{\cat{D}}$. Explicitly, this consists of the
	following data:
	\begin{itemize}
		\item An object $Fc\in \cat{D}$, for each object $c\in \cat{C}$.

		\item A morphism $\func{Ff}{Fc'}{Fc}\in \cat{D}$, for each morphism
			$\func{f}{c}{c'}\in \cat{C}$, so that the domain and codomain of
			$Ff$ are, respectively, equal to $F$ applied to the codomain or
			domain of $f$.
	\end{itemize}
	The assignments are required to satisfy the following two
	\emph{functoriality axioms}:
	\begin{itemize}
		\item For any composable pair $f$, $g$ in $\cat{C}$, $Ff Fg =
			F(gf)$.

		\item For each object $c$ in $\cat{C}$, $F(1_c) = 1_{Fc}$.
	\end{itemize}
\end{definition}

\settheorem{1}{3}{11}
\begin{definition}
	If $\cat{C}$ is locally small, then for any object $c\in \cat{C}$ we may
	define a pair of covariant and contravariant \emph{functors represented by}
	$c$:
	\[
	\xymatrix@-1pc{
	\cat{C}\ar[rr]^{\cat{C}(c,-)} & & \Set & \cat{C}^\op \ar^{\cat{C}(-,c)}[rr] & & Set \\
	x\ar[dd]_{f} & \mapsto & \cat{C}(c,x)\ar[dd]^{f_*} & x\ar[dd]_{f} & \mapsto & \cat{C}(x,c) \\
	  &\mapsto & & & \mapsto & &\\
	y & \mapsto & \cat{C}(c,y) & y & \mapsto & \cat{C}(y,c)\ar[uu]_{f^*} \\
	}
	\]
	The notation suggests the action on objects: the functor $\cat{C}(c,-)$
	carries $x\in\cat{C}$ to the set $\cat{C}(c,x)$ of arrows from $c$ to $x$ in
	$\cat{C}$. Dually, the functor $\cat{C}(-,c)$ carries $x\in\cat{C}$ to the
	set $\cat{C}(x,c)$.

	The functor $\cat{C}(c,-)$ carries a morphism $\func{f}{x}{y}$ to the
	post-composition function $\func{f_*}{\cat{C}(c,x)}{\cat{C}(c,y)}$
	introduced in Lemma 1.2.3(ii). Dually, the functor $\cat{C}(-,c)$ carries
	$f$ to the pre-composition function $\func{f^*}{\cat{C}(y,c)}{\cat{C}(x,c)}$
	introduced in 1.2.3(iii).
\end{definition}
\popthm

\begin{exercise}
	Verify that the constructions in Definition 1.3.11 are functorial.
\end{exercise}

\begin{proof}
	We start by showing that the assignments of $\cat{C}(c,-)$ satisfy the
	functoriality axioms for (covariant) functors. The actions of $\dom$ and
	$\cod$ on $\cat{C}(c,-)$ can be seen as follows: applying $\cat{C}(c,-)$ to
	a morphism $\func{h}{i}{j}$ will give a morphism
	$\func{\cat{C}(c,-)(h)}{\cat{C}(c,\dom h)}{ \cat{C}(c,\cod h)}$, so
	$\dom\cat{C}(c,-)(h) = \cat{C}(c,i)$ and $\cod\cat{C}(c,-)(h) =
	\cat{C}(c,j)$. \\

	To show composition, let $\func{f}{x}{y}$ and $\func{g}{w}{x}$ be a
	composable pair of morphisms in $\cat{C}$. Note first that $\func{f
	g}{w}{y}$ and that $\func{\cat{C}(c,-)(f)}{\cat{C}(c,x)}{\cat{C}(c,y)}$, and
	finally that $\func{\cat{C}(c,-)(g)}{\cat{C}(c,w)}{\cat{C}(c,x)}$. \\ Since
	$\dom\cat{C}(c,-)(f) = \cod\cat{C}(c,-)(g) = \cat{C}(c,x)$, we can compose
	as follows: \[\func{\cat{C}(c,-)(f)
	\qty{\cat{C}(c,-)(g)}}{\cat{C}(c,w)}{\cat{C}(c,y)}.\] Since
	$\cat{C}(c,-)(f g)$ 
	and $ \cat{C}(c,-)(f)\qty{\cat{C}(c,-)(g)} $ 
	are given by applying $f g$ to a morphism in
	$\cat{C}(c,w)$, we have that $\cat{C}(c,-)(f) \cat{C}(c,-)(g) =
	\cat{C}(c,-)(f g)$, satisfying functor composition.\\

	To show that identities are preserved, note that for any object
	$x\in\cat{C}$, $\func{1_x}{x}{x}$. Then
	$\func{\cat{C}(c,-)(1_x)}{\cat{C}(c,x)}{\cat{C}(c,x)}$ taking $1_x$ to the
	post composition $\func{1^{*}_{x}}{\cat{C}(c,x)}{\cat{C}(c,x)}$. Since for
	any morphism $a\in\cat{C}(c,x)$, $1^{*}_x$ takes $a\mapsto 1_xa$, this is
	the identity $a\mapsto a$. Then consider $1_{\cat{C}(c,-)(x)} =
	1_{\cat{C}(c,x)}$, which is the identity of $\cat{C}(c,x)$, taking each
	element $a$ of the set to itself: $a\mapsto a$. Thus, $\cat{C}(c,-)(1_x) =
	1_{\cat{C}(c,-)(x)}$ and $\cat{C}(c,-)$ preserves identities, as show by the following diagram.\\
	\[\xymatrix{
		c\ar[r]^a\ar[dr]_{1_xa}&x\ar[d]^{1_x} \\
		& y=x
	}
	\]

	To see that $\cat{C}(-,c)$ is a contravariant functor, we argue by duality.
	Since $\func{\cat{C}(c,-)}{\cat{C}}{\Set}$ is a functor for any category
	$\cat{C}$, we have that $\func{\cat{C}^\op(c,-)}{\cat{C}^\op}{\Set}$ is also
	a functor. Additionally, it is contravariant since the definition a
	contravariant functor is a functor from $\cat{C}^\op$ to $\Set$. Given that
	$\cat{C}^\op(c,-) = \cat{C}(-,c)$, we know that $\cat{C}(-,c)$ is a
	contravariant functor, completing the proof.
\end{proof}
\end{document}
