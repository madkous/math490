\documentclass[../../main]{subfiles}
\begin{document}

\paragraph{}
\begin{exercise}
	Given functors $\func{F}{\sD}{\sC}$ and $\func{G}{\sE}{\sC}$, show that
	there is a category, called the {\bf comma category} $F\downarrow G$, which
	has
	\begin{enumerate}
		\item as objects, triples $\qty{d \in \sD,e \in \sE, \func{f}{Fd}{Ge}
			\in \sC}$, and

		\item as morphisms $\qty{d,e,f}\rightarrow \qty{d',e',f'}$, a pair of
			morphisms $\qty{\func{h}{d}{d'},\func{k}{e}{e'}}$ so that the square
			\[\xymatrix{
					Fd \ar[d]_{Fh} \ar[r]^{f} & Ge\ar[d]^{Gk}  \\
			Fd'\ar[r]_{f'} &Ge'}\]
			commutes in $\sC$, i.e., so that $f' Fh = Gk  f$.
	\end{enumerate}
	Define a pair of projection functors $\func{\dom}{F\downarrow G}{\sD}$ and
	$\func{\cod}{F\downarrow G}{\sE}$
\end{exercise}

\begin{proof}
	Before we prove that the comma category $F\downarrow G$ is actually a
	category, we need to give a motivating example for a major issue in the
	proof.

	Let \(\func{A}{\mathsf{2}}{\mathsf{Set}}\) and
	\(\func{B}{\mathsf{2}}{\mathsf{Set}}\) be functors where $A0 = \{0\}$, $A1 =
	\{0,1,2\}$, $B0 = \{0,1\}$, $B1 = \{0,1,2,3\}$ and where $A$ and $B$ maps
	the unique morphism \(\func{f}{0}{1}\) to the inclusion functions
	\(\func{\io}{\{0\}}{\{0,1,2\}}\)  and \(\func{\io}{\{0,1\}}{\{0,1,2,3\}}\)
	respectively. Let us take our supposed objects $\qty{0,0,\io}$  and
	$\qty{1,1,\al}$ where $\io$ is the inclusion function and our supposed
	morphism $\qty{\func{f}{0}{1},\func{f}{0}{1}}$ so that the diagram
	\[\xymatrix{ &A0 \ar[d]_{Af} \ar[r]^{\io} & B0\ar[d]^{Bf}  \\  &
	A1\ar[r]_{\al} &B1   } \] commutes. Now there are at least two functions for
	$\al$ that would allow the diagram above to commute. The first is if $\al$
	was simply an inclusion function so the functions $\al  Af$ and $Bf
	 \io$ are inclusion functions from the singleton set $A0$ to $B1$, thus
	$\al  Af$ =  $Bf  \io$. The second function which I will denote
	$\al'$ is defined as follows:\[ \al'(0) = 0, \al'(1) = 2,\al'(2) = 1.\]
	Since $\al'$ still maps $0$ in $A1$ to itself in $B1$, the diagram above
	still commutes. Thus our supposed morphism
	$\qty{\func{f}{0}{1},\func{f}{0}{1}}$ would have two codomains
	($\qty{1,1,\al}$ and $\qty{1,1,\al'}$) for our domain $\qty{0,0,\io}$.

	This example shows we need additional notation to distinguish between arrows
	that are represented the same but have different domains and codomains.
	Returning to the notation established in the first paragraph, we will append
	morphism pairs $\qty{f,f'}$ to the end of some morphism
	$\qty{\func{h}{d}{d'},\func{k}{e}{e'}}$ so that we specify that the intended
	domain and codomain of the morphism is $\qty{d,e,f}$ and $\qty{d',e',f'}$
	respectively. Thus the uniqueness of the domain and codomain of some
	morphism $\qty{\func{h}{d}{d'},\func{k}{e}{e'}} \qty{f,f'}$ follows from the
	uniqueness of the domain and codomain of $h$ and $k$, and additionally from
	our notation which specifies unique morphisms $f$ and $f'$. Now we can
	complete the rest of the proof using the notation established in the first
	paragraph.

	For an object $\qty{d,e,f}$,denoted as $c$, we can define an identity
	morphism for $c$ as the following:  \[ 1_c  = \qty{1_{d},1_{e}}\qty{f,f}\]
	where $1_{d}$ and $1_{e}$ are the respective identities of $d$ and $e$. Thus
	the diagram \[\xymatrix{ &Fd \ar[d]_{F1_{d}} \ar[r]^{f} & Ge\ar[d]^{G1_{e}}
						 \\  & Fd\ar[r]_{f} &Ge   } \] trivially commutes. The
		unique domain and codomain of $1_c$, both being $\qty{d,e,f}$, are
		derived from the unique domain and codomains of the identities $1_d$ and
		$1_e$, and the uniquely specified $f$.

	Now let us define morphism composition between two morphisms.
	$$\qty{\func{h}{d}{d_1},\func{k}{e}{e_1}} \qty{f,f_1} \AND
	\qty{\func{h}{d_1}{d_2},\func{k}{e_1}{e_2}} \qty{f_1,f_2}$$ which we will
	denote $$\func{\al}{(d,e,f)}{(d_1,e_1,f_1)} \AND
	\func{\be}{(d_1,e_1,f_1)}{(d_2,e_2,f_2)}.$$
	The composition of $\al$ and $\be$ shall be defined as follows:
	\[\be  \al=\qty{\func{h'  h}{d}{d_{2}},\func{k'  k}{e}{e_{2}}}\qty{f,f_2}\]
	resulting in the diagram
	\[\xymatrix{ Fd \ar[d]_{Fh} \ar[r]^{f} & Ge\ar[d]^{Gk}  \\
			Fd_{1} \ar[d]_{Fh'}\ar[r]^{f_{1}}& Ge_{1} \ar[d]^{Gk'} \\
	Fd_{2} \ar[r]^{f_{2}} &Ge_2 } \]
	which commutes since functors preserve composition of morphisms and the top
	and bottom squares commute by construction of $\al$ and $\be$\footnote{See diagram 1.6.10.}. Thus we have
	that the following diagram commutes:
	\[\xymatrix{ &Fd \ar[d]_{Fh'  Fh}\ar[r]^{f} & Ge\ar[d]^{Gk' Gk}
			 \\  & Fd_2\ar[r]^{f_2} &Ge_2 }\]
	which is the commutative square for the composed morphism $\be  \al$.
	The morphism $\be \al$ derives its unique domain from the unique
	domains of $h' h$ and $k' k$ and the specified function $f$
	which gives the domain $\qty{d,e,f}$ which is the domain of $\al$, and the
	unique codomain is derived similarly resulting in the codomain
	$\qty{d_2,e_2,f_2}$ which is the codomain of $\be$. So the composition of
	morphisms $\al$ and $\be$ gives a morphism $\be \al$ with the domain
	of $\al$ and the codomain of $\be.$

	Now that we have defined the identity morphism and composition of morphism,
	we can show that the identity morphism is left and right cancellable, and
	that composition is associative.

	Let $\qty{\func{h}{d}{d'},\func{k}{e}{e'}} \qty{f,f'}$ , denoted as $\al$,
	be a morphism with domain and codomain $\qty{d,e,f}$ and $\qty{d',e',f',}$
	respectively, denoted as $c$ and $c'$ respectively. Starting with the
	composition of $\al$ and $1_{c}$, we can show the following chain of
	equalities:
	\begin{align*}
	\al  1_{c } &= \qty{h 1_{d},k 1_{e}}\qty{f,f'}\\
	&= \qty{h,k}\qty{f,f'} \\
	 &= \al.
	\end{align*}
	Composing  $1_{c'}$ and $\al$ gives us a similar result:
	\begin{align*}
	1_{c' } \al &= \qty{1_{d'} h, 1_{e'} k}\qty{f,f'}\\
	&= \qty{h,k}\qty{f,f'}\\
	&= \al.
	\end{align*}
	Thus we have shown
	that $1_{c' } \al = \al = \al 1_{c}$. Therefore the identity
	morphism is left and right cancellable.

	Finally, we will show that the composition of morphisms is associative. Take
	morphisms
	$\qty{\func{h}{d}{d_1},\func{k}{e}{e_1}} \qty{f,f_1}$,
	$\qty{\func{h}{d_1}{d_2},\func{k}{e_1}{e_2}} \qty{f_1,f_2}$, and
	$\qty{\func{h}{d_2}{d_3},\func{k}{e_2}{e_3}} \qty{f_2,f_3}$, denoted as
	$\al$, $\be$, and $\gm$ respectively. Now observe that:
	\begin{align*}
	(\gm\be)\al&=((h_2h_1),(k_2 k_1))(f_1,f_3) \al\\
	&=((h_2 h_1) h,(k_2 k_1) k)(f,f_3)\\
	&=(h_2(h_1 h),k_2(k_1 k))(f,f_3)\\
	&=\gm((h_1 h),(k_1 k))(f,f_2)\\
	&=\gm(\be \al).
	\end{align*}
	Thus the composition of morphisms is associative, and we have shown
	that $F \downarrow G$ is a category.

	Now we will define the functors $\func{\dom}{F \downarrow G}{\sD}$ and
	$\func{\cod}{F \downarrow G}{\sE}$ for object $\qty{d,e,f}$ and morphism
	$\qty{\func{h}{d}{d'},\func{k}{e}{e'}} \qty{f,f'}$ as follows:
	\begin{align*}
	\dom\qty{d,e,f} &= d, \dom\qty{h,k}\qty{f,f'} \;\;= h \\
	\cod\qty{d,e,f} &= e, \cod\qty{h,k}\qty{f,f'} \quad= k.
	\end{align*}

	Now we will verify that both $\dom$ and $\cod$ are indeed functors.

	Now let us take the morphism $\qty{\func{h}{d}{d_1},\func{k}{e}{e_1}}
	\qty{f,f_1}$  denoted $\func{\al}{\qty{d,e,f}}{\qty{d_1,e_1,f_1}}.$
	Applying $\dom$ to $\al$ gives us $\func{h}{d}{d_{1}}$ where $\dom\qty{d,e,f}
	= d$ and $\dom\qty{d_1,e_1,f_1} = d_1$. Applying $\cod$ to $\al$ gives
	$\func{k}{e}{e_{1}}$ where $\cod\qty{d,e,f} = e$ and $\cod\qty{d_1,e_1,f_1} =
	e_1$. Thus $\func{\al}{\qty{d,e,f}}{\qty{d_1,e_1,f_1}}$ gets mapped to
	$\func{\dom\al}{\dom\qty{d,e,f}}{\dom\qty{d_1,e_1,f_1}}$ and
	$\func{\cod\al}{\cod\qty{d,e,f}}{\cod\qty{d_1,e_1,f_1}}$ by $\dom$ and $\cod$
	respectively.

	For object $\qty{d,e,f}$, the identity arrow $\qty{1_{d},1_{e}}\qty{f,f}$
	get mapped to $1_d$ and $1_e$ by $\dom$ and $\cod$ respectively. Since
	$\dom\qty{d,e,f} = d$ and $\cod\qty{d,e,f} = e$, This shows that $\dom$ and
	$\cod$ preserve identities.

	Finally take morphisms $\qty{\func{h}{d}{d_1},\func{k}{e}{e_1}} \qty{f,f_1}$
	denoted $\func{\al}{\qty{d,e,f}}{\qty{d_1,e_1,f_1}}$ and
	$\qty{\func{h'}{d_1}{d_2},\func{k'}{e_1}{e_2}} \qty{f_1,f_2}$   denoted
	$\func{\be}{\qty{d_1,e_1,f_1}}{\qty{d_2,e_2,f_2}}$. Then we have the
	following equalities:
	\[\dom(\be \al) = \dom\qty{h'  h,k'  k}\qty{f,f_2} =h'
	 h = \dom\be  \dom\al\] and
	\[\cod(\be \al) = \cod \qty{h' h,k' k}\qty{f,f_2}=  k'
	 k = \cod\be\cod\al.\] Thus $\dom$ and $\cod$ perserve morphism
	composition, and $\dom$ and $\cod$ are functors.
\end{proof}
\end{document}

