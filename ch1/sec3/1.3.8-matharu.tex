\documentclass[main.tex]{subfiles}
\begin{document}

\paragraph{}
\setexercise{1}{3}{8}
\begin{exercise}
	Lemma 1.3.8 shows that functors preserve isomorphisms. Find an example to
	demonstrate that functors need not \textbf{reflect isomorphisms}: that is,
	find a functor $ F:\sC\to\sD $ and a morphism $ f\IN\sC $ so that $ Ff $ is
	an isomorphism in $ \sD $ but $ f $ is not an isomorphism in $ \sC. $
\end{exercise}


Consider the functor $ F: \mathbb{2}\to \mathbb{1}.$ This is a great functor
because it maps everything in $\mathbb{2}$ to the identity in $ \mathbb{1}. $ So
$ F $ trivially satisfies all the properties of a functor. Because $ \mathbb{2}
$ is not a groupoid there exists at least one morphism $ f $ in $\mor \mathbb{2}
$ that is not an isomorphim. In this construction $ Ff $ will also go to the
best isomorphism the identity map.
\end{document}
