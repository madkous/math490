\documentclass[../../main]{subfiles}
\begin{document}

\paragraph{}
\begin{exercise}
	Generalize Proposition 3.3.8 to show that for any $F:\sC\rightarrow\Set$, the projection functor $\Pi:\el F\rightarrow\sC$:
	\begin{enumerate}
		\item strictly creates all limits that $\sC$ admits and that $F$ preserves, and
		\item strictly creates all connected colimits that $\sC$ admits.
	\end{enumerate}
\end{exercise}

Consider an ordered pair $(K,x)$ where $K:J\rightarrow\sC$ and $x:1\Rightarrow
FK$. Here $1=\{0\}$ is a singleton set, or the corresponding constant functor
from $J$ to $\Set$, and we identify $x_i:1\rightarrow FKi$ with $x_i(0)\in FKi$.
This pair gives a diagram $(K,x):J\rightarrow\el F$. For each morphism
$g:i\rightarrow j$ in $J$ the morphism $(K,x)g:(Ki,x_i)\rightarrow(Kj,x_j)$ is
given by $Kg:Ki\rightarrow Kj$. This is a morphism in $\el F$ because the
natural transformation $x:1\Rightarrow FK$ tells us that $FKg(x_i)=x_j$.

Now, all diagrams $L:J\rightarrow\el F$ are of this form. Indeed, let $K=\Pi L$.
Then for every $i\in\ob J$ we have that $Li=(Ki,x_i)$ for some $x_i\in FKi$ and
for every $g:i\rightarrow j$, $Lg:(Ki,x_i)\rightarrow(Kj,x_j)$ is given by
$Kg:Ki\rightarrow Kj$ such that $FK(x_i)=x_j$. But, this precisely gives a
natural transformation $x:1\Rightarrow FK$ as described above.

\subparagraph{}
Now, for part (1) say that $K:\sJ\rightarrow\sC$ has a limit cone $\lambda:\lim
K\Rightarrow K$ and that $F$ preserves this limit. That is, $F\lambda:F\lim
K\Rightarrow FK$ is also a limit cone. Then the natural transformation
$x:1\Rightarrow FK$ must uniquely factor through $F\lambda$. That is, there is a
unique $\hat{x}:1\rightarrow F\lim K$ such that $F\lambda\circ\hat{x}=x$. We
conflate the symbol $\hat{x}$ with $\hat{x}(0)\in F\lim K$. By construction,
$F\lambda_i(\hat{x})=x_i\in Fi$. This gives us a cone $\hat{\lambda}:(\lim
K,\hat{x})\Rightarrow(K,x)$. Furthermore, $\Pi\hat{\lambda}=\lambda$.

We must now show that $\hat{\lambda}$ is the only lift of $\lambda$ and that
$\hat{\lambda}$ is a limit cone. So, let $\mu:(c,z)\Rightarrow(K,x)$ be a
natural transformation for which $(c,z)\in\ob\el F$ (that is, $c\in\ob\sC$ and
$z\in Fc$) and $\Pi\mu=\lambda$. Then $c=\Pi(c,z)=\lim K$ so that $z\in F\lim K$
and $F\lambda_i(z)=x_i$ for each $i\in\ob\sJ$. Identifying $z$ with the constant
function $z:1\rightarrow F\lim K$ we have that $F\lambda\circ z=x$. But,
$\hat{x}$ was uniquely defined by this property, so that $z=\hat{x}$. Comparing
now with $\hat{\lambda}$, we have that $\mu=\hat{\lambda}$ as required. This
proves that $\hat{\lambda}$ is the unique lift of $\lambda$.

Now, we must see that $\hat{\lambda}:(\lim K,\hat{x})\Rightarrow(K,x)$ is a
limit cone. Let $\mu:(c,z)\Rightarrow(K,x)$ be another cone. (Here,
$c\in\ob\sC$, $z\in Fc$ and $\mu$ are not the same $c$, $z$ and $\mu$ as in the
previous paragraph.) We must prove that $\mu$ factors uniquely through
$\hat{\lambda}$. Since $\Pi\mu:c\Rightarrow K$, there is a unique
$f:c\rightarrow\lim K$ in $\sC$ such  that $\lambda f=\Pi\mu$. Also, for every
$i\in\ob J$, $x_i=F\Pi\mu_i(z)=F\lambda_i f(z)$. Just as in the previous
paragraph, we have that $F\lambda\circ f(z)=x$ so that $f(z)$ satisfies the
defining property of $\hat{x}$. Thus, $f(z)=\hat{x}$ and $f:c\rightarrow\lim K$
determines a morphism $\hat{f}:(c,z)\rightarrow(\lim K,\hat{x})$ in $\el F$.
This morphism is the unique morphism such that $\hat{\lambda}\hat{f}=\mu$,
showing that $\hat{\lambda}:(\lim K,\hat{x})\Rightarrow(K,x)$ is a limit cone.

\subparagraph{}
For the second part, say that $\sJ$ is a small connected category (which is in particular nonempty) and that $(K,x):\sJ\rightarrow\el F$ is a diagram such that $K:\sJ\rightarrow\sC$ has a colimit cone $\lambda:K\Rightarrow\colim K$. We start by building a lift $\hat\lambda:(K,x)\Rightarrow(\colim K,\hat x)$. Since $x:1\Rightarrow FK$ is a natural transformation, for any $g:i\rightarrow j$ in $\sJ$ we have that $x_j=FKg(x_i)$. Using that $\lambda:K\Rightarrow\colim K$ is a natural transformation and $F$ is a functor, we find that
\[F\lambda_i(x_i)=F(\lambda_jKg)(x_i)=F\lambda_jFKg(x_i)=F\lambda_j(x_j).\]
Since $\sJ$ is connected, this implies that $F\lambda_i(x_i)=F\lambda_j(x_j)$ for arbitrary objects $i,j\in\ob\sJ$. Since $\sJ\ne\emptyset$, we may then define $\hat{x}\in F\colim\sJ$ by $\hat{x}=\lambda_i(x_i)$ for an arbitrary choice of $i\in\ob\sJ$ and the definition is independent of that choice. This gives a cone $\hat{\lambda}:(K,x)\Rightarrow(\colim K,\hat x)$ under $(K,x)$ such that $\Pi\hat{\lambda}=\lambda$.

We now show that $\hat{\lambda}$ is the unique lift of $\lambda$. Let
$\mu:(K,x)\Rightarrow(c,z)$ be another cone under $(K,x)$ such that
$\Pi\mu=\lambda$. Then $c=\colim K$ since $\Pi\mu=\lambda$. Furthermore, for any
$i\in\ob\sJ$, $z=F\lambda_i(x_i)=\hat{x}$. So, $\mu$ is identical with
$\hat\lambda$.

Finally, we must show that $\hat\lambda:(K,x)\Rightarrow(\colim K,\hat x)$ is a colimit cone. Let $\mu:(K,x)\Rightarrow(c,z)$ be another cone under $(K,x)$. Then $\Pi\mu:K\Rightarrow c$ is a cone under $K$ in $\sC$, so that there is a unique morphism $f:\colim K\rightarrow c$ in $\sC$ such that $\Pi\mu=f\lambda$. Further, for any $i\in\ob\sJ$, we have that
\[Ff(\hat{x})=FfF\lambda_i(x_i)=F(f\lambda_i)(x_i)=F\Pi\mu_i(x_i)=z.\]
So, the $f$ that we uniquely determined gives a morphism $\hat f:(\colim K,\hat x)\rightarrow(c,z)$, the unique morphism such that $\mu=\hat f\hat\lambda$. This proves that $\hat{\lambda}:(K,x)\Rightarrow(\colim K,\hat x)$ is a colimit cone.
\end{document}
