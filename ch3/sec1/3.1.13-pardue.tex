\documentclass[../../main]{subfiles}
\begin{document}

\begin{exercise}
	What is the coproduct in the category of commutative rings?
\end{exercise}

The solution for this exercise requires tensor products of abelian groups. For
the full solution, we will need to consider tensor products of arbitrary
families of abelian groups. In writing this solution, I expand on ideas from the
end of Chapter 2 of {\it Introduction to Commutative Algebra} by Atiyah and
MacDonald, and Appendix A6.3 of {\it Commutative Algebra with a View Toward
Algebraic Geometry} by Eisenbud.

In principal, we could just dive in and construct the coproduct of an arbitrary
family of commutative rings. But, it is best to see how this works in the case
of two commutative rings first.

So, in the first part of this solution we will  take the following steps:
\begin{enumerate}
	\item Describe the tensor product $A\otimes B$ of two abelian  groups $A$  and $B$.
	\item Show that for two commutative rings $R$ and $S$, the tensor product $R\otimes S$ of their additive groups has a well-defined multiplication law, making $R\otimes S$ a commutative ring. This is the  trickiest part.
	\item Describe ring homomorphisms from $R$ and  $S$ to $R\otimes S$ and show that they satisfy the universal property  for the coproduct in the category of commutative rings.
\end{enumerate}

After all that hard work, we will do it over again for arbitrary families of
abelian groups and arbitrary families of commutative rings.

Riehl introduces tensor products of $\mbb{k}$-vector spaces in Example 2.3.7 and
continues their discussion through Remark 2.3.11 and Exercise 2.3.ii. Although
she does not show that tensor products exist (they do), she does derive a number
of basic properties.

The entire discussion remains the same if $\mbb{k}$-vector spaces are replaced
by $R$-modules, where $R$ is a commutative ring.\footnote{Tensor products of
modules for noncommutative rings are trickier and do not  exist in the
generality that we will discuss  in the second half. See Jacobson's {\it Basic
Algebra II} for basic information on tensor products of modules over
noncommutative rings.} Since we are only concerned with abelian groups, and
abelian groups are $\ZZ$-modules, we may take $R=\ZZ$. This simplifies a few
points, though the general case is not much harder.

\subparagraph{}
For convenience, we will write all abelian groups additively. If $C$ is a third abelian group, then a bilinear function
\[f:A\times B\rightarrow C\]
is as defined in Example 2.3.7 with "linear map" replaced by "group homomorphism". That is, we require that $f(a_1+a_2,b)=f(a_1,b)+f(a_2,b)$ and $f(a,b_1+b_2)=f(a,b_1)+f(a,b_2)$. It follows for any $n\in\ZZ$ that $f(na,b)=nf(a,b)=f(a,nb)$. If also $g:C\rightarrow D$ is a homomorphism of abelian groups, then it is easy to check that $gf:A\times B\rightarrow D$ is bilinear. So, we have a functor
\[\text{Bilin}(A,B;-):\Ab\rightarrow\Set\]
where $\text{Bilin}(A,B;C)$ is the set of bilinear functions $f:A\times B\rightarrow C$. If $g:C\rightarrow D$ is as above, then post-composition gives a function $g_*:\text{Bilin}(A,B;C)\rightarrow\text{Bilin}(A,B;D)$, making $\text{Bilin}(A,B;-)$ a functor. Any abelian group representing this functor is the \emph{tensor product} of $A$ and $B$ and has earned the right to be written as $A\otimes B$.

We now give a standard construction of the tensor product of two abelian groups $A$ and $B$. Let $F$ be the free abelian group whose basis is the set $A\times B$. That is, each element of $F$ may be uniquely written as $\sum _{i=1}^nm_i(a_i,b_i)$ where $n\in\NN$, $m_i\in\ZZ$ and $(a_1,b_1),\dots ,(a_n,b_n)$ are distinct elements of $A\times B$. Let $K\subseteq F$ be the subgroup generated by
\[(a_1+a_2,b)-(a_1,b)-(a_2,b)\text{ and }(a,b_1+b_2)-(a,b_1)-(a,b_2)\]
for $a_1,a_2,a\in A$ and $b_1,b_2,b\in B$. The generators are precisely what we need to guarantee that
\[(a_1+a_2,b)\equiv(a_1,b)+(a_2,b)\text{ and } (a,b_1+b_2)\equiv(a,b_1)+(a,b_2)\]
modulo $K$. It follows that for each generator $(a,b)$ of $F$ and each integer $m\in\ZZ$,
\[(ma,b)\equiv m(a,b)\equiv (a,mb)\text{ modulo }K.\]

Write $A\otimes B$ for $F/K$. Write $a\otimes b$ for the congruence class of $(a,b)$ in $A\otimes B$. Then we immediately have the following identities in $A\otimes B$, with $a,a_1,a_2,b,b_1,b_2,m$ as above:
\[
	(a_1+a_2)\otimes b = a_1\otimes b+a_2\otimes b, \,\,\,\,
	a\otimes(b_1+b_2) = a\otimes b_1+a\otimes b_2,\,\,\,\,
	(ma)\otimes b=m(a\otimes b)=a\otimes(mb).
\]

It follows that each element of $A\otimes B$ has a (non-unique) representation as
\[\sum _{i=1}^na_i\otimes b_i.\]
An element of $A\otimes B$ of the form $a\otimes b$ is called a \emph{simple tensor}.

Let $\otimes:A\times B\rightarrow A\otimes B$ be the function given by $\otimes(a,b)=a\otimes b$. Then the identities above show that $\otimes$ is a bilinear function. It follows that for any morphism $g:A\otimes B\rightarrow C$ in $\Ab$, $g\circ\otimes:A\times B\rightarrow C$ is also bilinear. This is one direction of a natural bijection
\[\text{Bilin}(A,B;C)\leftrightarrow\Ab(A\otimes B,C).\]
For the other direction, let $f:A\times B\rightarrow C$ be a bilinear function. I claim that there is a unique group homomorphism $\bar{f}:A\otimes B\rightarrow C$ such that $f=\bar{f}\circ\otimes$. For such an $\bar{f}$ to exist, it would need to satisfy for every $a\in A$ and $b\in B$ that
\[f(a,b)=\bar{f}\circ\otimes(a,b)=\bar{f}(a\otimes b),\]
so there is at most one. To see that $\bar{f}$ does exist, first define a group homomorphism $\tilde{f}:F\rightarrow C$ by $\tilde{f}((a,b))=f(a,b)$. Such a homomorphism exists because $F$ is free with basis consisting of all of the $(a,b)\in A\times B$. Since $f$ is bilinear, each generator of $K$ is in $\ker\tilde{f}$. So, $\tilde{f}:F\rightarrow C$ factors through $F/K$, and the induced map $\bar{f}:F/K\rightarrow C$ is $\bar{f}(a\otimes b)=f(a,b)$ as required.

\subparagraph{}
Now, what does this have to with the problem at hand? We will construct the coproduct of two commutative rings $R$ and $S$ as a commutative ring whose additive group is $R\otimes S$. While the multiplication law on $R\otimes S$ will seem obvious, the trick will be in seeing that it is well-defined.

We now carry out this construction. Let $R$ and $S$ be our rings and $R\otimes S$ be the tensor product of $(R,+)$ and $(S,+)$. Choose any $(r_1,s_1)\in R\times S$. Then we have a function
\[\mu_{(r_1,s_1)}:R\times S\rightarrow R\otimes S\]
taking $(r_2,s_2)\in R\times S$ to $r_1r_2\otimes s_1s_2$. Using the distributive laws for $R$ and $S$ and the identities satisfied by $\otimes$, we see that $\mu_{(r_2,s_2)}$ is bilinear:
\[\begin{array}{rcl}\mu_{(r_1,s_1)}(r_2+r_3,s_2) &=& r_1(r_2+r_3)\otimes s_1s_2\\&=&(r_1r_2+r_1r_3)\otimes s_1s_2\\&=&r_1r_2\otimes s_1s_2+r_1r_3\otimes s_1s_2\\
												 &=&\mu_{(r_1,s_1)}(r_2,s_2)+\mu_{(r_1,s_1)}(r_3,s_2).\end{array}\]
		Similarly, $\mu_{(r_1,s_1)}(r_2,s_2+s_3)=\mu_{(r_1,s_1)}(r_2,s_2)+\mu_{(r_1,s_1)}(r_2,s_3)$. So, $\mu_{(r_1,s_1)}$ factors through a group homomorphism $R\otimes S\rightarrow R\otimes S$ taking $r_2\otimes s_2$ to $r_1r_2\otimes s_1s_2$.

		Thus, we have a function $\mu:R\times S\times(R\otimes S)$ defined by $\mu(r_1,s_1,r_2\otimes s_2)=r_1r_2\otimes s_1s_2$. Arguing as above, we fix $r_2\otimes s_2\in R\times S$. Then $\mu(-,-,r_2\otimes s_2):R\times S\rightarrow R\otimes S$ is bilinear and thus factors through a homomorphism $R\otimes S\rightarrow R\otimes S$ taking $(r_1,s_1)$ to $r_1r_2\otimes s_1s_2$. Thus, $\mu$ induces a well-defined function
		\[\bar{\mu}:(R\otimes S)\times(R\otimes S)\rightarrow R\otimes S\]
		that on simple tensors is given by $\bar{\mu}(r_1\otimes s_1,r_2\otimes s_2)=r_1r_2\otimes s_1s_2$. Furthermore, $\bar{\mu}$ is bilinear!

		$\bar{\mu}$ is the multiplication of simple tensors on our putative ring $R\otimes S$. An arbitrary product is now given by
		\[\left(\sum _{i=1}^mr_{1i}\otimes s_{1i}\right)\left(\sum _{j=1}^nr_{2j}\otimes s_{2j}\right)=\sum _{i=1}^m\sum _{j=1}^nr_{1i}r_{2j}\otimes s_{1i}s_{2j}.\]

		Multiplication in $R\otimes S$ clearly distributes over addition in $R\otimes S$. It follows that multiplication is associative. Indeed the product of the three elements
		\[\sum _{i=1}^mr_{1i}\otimes s_{1i},\,\,\,\sum _{j=1}^nr_{2j}\otimes s_{2j}\text{ and }\sum _{k=1}^pr_{3k}\otimes s_{3k}\]
		is equal to
		\[\sum _{i=1}^m\sum _{j=1}^n\sum _{k=1}^pr_{1i}r_{2j}r_{3k}\otimes s_{1i}s_{2j}s_{3k}\]
		no matter in what order the products are computed. The identity is $1=1\otimes 1$, so that $R\otimes S$ is a monoid under multiplication. This is enough to see that $R\otimes S$ is a ring. That $R\otimes S$ is a commutative ring then follows from $R$ and $S$ being commutative.

		Now, we must see that $R\otimes S$ has the right property to be a coproduct of $R$ and $S$. First, consider the function $\iota_0:R\rightarrow R\otimes S$ given by $\iota_0(r)=r\otimes 1$. Then
		\[\begin{array}{ll}
				& \iota_0(r_1+r_2)=(r_1+r_2)\otimes 1=r_1\otimes 1+r_2\otimes 1=\iota_0(r_1)+\iota_0(r_2),\\
				& \iota_0(r_1r_2)=(r_1r_2)\otimes 1=(r_1\otimes 1)(r_2\otimes 1)=\iota_0(r_1)\iota_0(r_2),\\
				& \iota_0(1)=1\otimes 1=1.
			\end{array}
		\]
		So, $\iota_0$ is a ring homomorphism. Similarly, we have a ring homomorphism $\iota_1:S\rightarrow R\otimes S$ given by $\iota_1(s)=1\otimes s$.

		\subparagraph{}
		Finally, we must check the universal property. Let $T$ be a third commutative ring and let $f:R\rightarrow T$ and $g:S\rightarrow T$ be ring homomorphisms. Considering $R$, $S$ and $T$ under addition, the function $(f*g):R\times S\rightarrow T$ defined by $(f*g)(r,s)=f(r)g(s)$ is bilinear, as may be readily checked from the distributive law in $T$ and that $f$ and $g$ are homomorphisms for $+$. So, $f*g$ factors through a homomorphism $h:R\otimes S\rightarrow T$ on the additive groups given on simple tensors by $h(r\otimes s)=f(r)g(s)$.  It is readily seen from the commutativity of $T$ that this is also a homomorphism for multiplication, so that $h$ is a ring homomorphism. Since $h\iota_0(r)=h(r\otimes 1)=f(r)g(1)=f(r)$ and $h\iota_1(s)=h(1\otimes s)=f(1)g(s)=g(s)$, $R\otimes S$ satisfies the universal property of the coproduct in the category of commutative rings.

		\subparagraph{}
		Now, let's do it again for arbitrary families of commutative rings. We will perform the following steps.
		\begin{enumerate}
			\item Describe the tensor product $\bigotimes_{j\in J}A_j$ of an arbitrary family of abelian groups $(A_j)_{j\in J}$.
			\item Show that for a family of commutative rings $(R_j)_{j\in J}$, the tensor product $R=\bigotimes_{j\in J}R_j$ of their additive groups admits a multiplication law making $R$ into a commutative ring. This is a little tricky, but is similar to the case of the tensor product of two rings above.
			\item Describe a natural family of ring homomorphisms $R_j\rightarrow R$ that will make $R$ the coproduct if $J$ is finite.
			\item Describe a subring $R^\prime$ of $R$ that is the coproduct of the $R_j$ even when $J$ is not finite. This part requires some care, but is reminiscent of the direct sum (coproduct) of a family of modules being a submodule of the direct product (product) of that same family.
		\end{enumerate}

		Let $J$ be a set, viewed as a small discrete category, and consider a family $A_j$ of abelian groups indexed by $j\in J$. Let
		\[f:\prod_{j\in J}A_j\rightarrow C\]
		be a function to an abelian group $C$, where $\prod_{j\in J}A_j$ is the Cartesian product of the sets $A_j$.  Consider some $i\in J$ and a choice of fixed values $a_j\in A_j$ for $j\ne i$ and let $\alpha:A_i\rightarrow\prod_{j\in J}A_j$ be the function such that $\alpha(a)=(a_j)_{j\in J}$ where $a_j=a$ if $j=i$, and is the chosen fixed value otherwise. If for every choice of $i\in J$ and fixed values $a_j\in A_j$ for $j\ne i$ the composition $f\alpha:A_i\rightarrow C$ is a group homomorphism, then $f$ is called a \emph{$J$-linear function}.

		Before going on, let's check what happens when $J$ has no more than $2$ elements. If $J=\emptyset$, then $\prod_{j\in\emptyset}A_j$ is a singleton set -- a terminal object in $\Set$. The conditions above on $f$ are vacuous, so we simply require a function $f:\{*\}\rightarrow C$, which is determined by a choice of element $c\in C$. If $J$ is a singleton set, then our product is just an abelian group $A_i=A$. There are no other $j\in J$, so that the choice of fixed values $a_j\in A_j$ is vacuous so that the condition on $f:A\rightarrow C$ is that it is a group homomorphism. If $J$ has two elements, then the condition of being a $J$-linear function is the condition of being a bilinear function considered above.

		\subparagraph{}
		Just as for bilinear functions, we have for a $J$-linear function $f:\prod_{j\in J}A_j\rightarrow C$ and a group homomorphism $g:C\rightarrow D$ that $gf:\prod_{j\in J}A_j\rightarrow D$ is also $J$-linear. Thus, for a $J$-indexed family of abelian groups $(A_j)_{j\in J}$ we have a functor
		\[J\text{-lin}((A_j)_{j\in J};-):\Ab\rightarrow\Set\]
		taking $C$ to the set of $J$-linear functions from $\prod_{j\in J}A_j$ to $C$ and taking a group homomorphism $g:C\rightarrow D$ to postcomposition $g_*$. Any group representing this functor is the \emph{tensor product} of this family of groups $(A_j)_{j\in J}$ and is denoted
		\[\bigotimes_{j\in J}A_j.\]
		Let's follow up on the basic cases above. If $J=\emptyset$, then $\emptyset\text{-lin}(();-)$ is naturally isomorphic to the forgetful functor $U:\Ab\rightarrow\Set$, which is represented by $\ZZ$. So, an empty tensor product is equal to $\ZZ$. If $J$ is a singleton set, then $1\text{-lin}(A;-)\simeq\Ab(A,-)$ which is tautologically represented by $A$. So, a tensor product of a single abelian group is that abelian group. The tensor product of two abelian groups is as we considered above.

		\subparagraph{}
		Now, we imitate the construction of a tensor product of two abelian groups to give the tensor product of an arbitrary family of abelian groups. Given a family $(A_j)_{j\in J}$ of abelian groups, let $F$ be the free abelian group whose basis is the elements of $\prod_{j\in J}A_j$. Thus, each element of $F$ may be uniquely expressed as $\sum _{i=1}^nm_i(a_{ij})_{j\in J}$ where $n\in\NN$, $m_i\in\ZZ$ and $(a_{1j})_{j\in J},\dots ,(a_{nj})_{j\in J}$ are distinct elements of $\prod_{j\in J}Aj$.

		Let $K\subseteq F$ be the subgroup generated by all expressions of the form
		\[(a_j)_{j\in J}-(b_j)_{j\in J}-(c_j)_{j\in J}\]
		such that there is some $i\in J$ for which $a_i=b_i+c_i$ and $a_j=b_j=c_j$ for all $j\ne i$. Write $\bigotimes_{j\in J}A_j$ for $F/K$ and $\otimes_{j\in J}a_j$ for the image of the basis element $(a_j)_{j\in J}$ in $\bigotimes_{j\in J}A_j$. This image is called a \emph{simple tensor}. So long as $J\ne\emptyset$, any integer multiple of a simple tensor $m(\otimes_{j\in J}a_j)$ may also be written as a simple tensor by choosing some $j$ and replacing $a_j$ by $ma_j$. Then, for $J\ne\emptyset$, each element of $\bigotimes_{j\in J}A_j$ may be represented (non-uniquely)  as a finite sum of simple tensors\footnote{If $J=\emptyset$, then every element of $\bigotimes_{j\in\emptyset}A_j$ is an integer multiple of the unique empty simple tensor. We will assume that $J\ne\emptyset$ going forwards, but the modifications required for the empty case are straightforward.}:
		\[\sum _{i=1}^n\otimes_{j\in J}a_{ij}.\]

		The choice of $K$ makes it so that the function
		\[\otimes:\prod_{j\in J}A_j\rightarrow\bigotimes_{j\in J}A_j\]
		taking $(a_j)_{j\in J}$ to $\otimes_{j\in J}a_j$ is $J$-linear. If $f:\prod_{j\in J}A_j\rightarrow C$ is another $J$-linear function to an abelian group $C$, then we wish to produce a homomorphism $\bar{f}:\bigotimes A_j\rightarrow C$ such that $f=\bar{f}\circ\otimes$. The only possibility is for $\bar{f}(\otimes a_j) = f(\otimes(a_j))$. We must see that this is a well-defined group homomorphism.

		Using that $F$ is free on a basis consisting of  $(a_j)_{j\in J}$, we have a unique group homomorphism $\tilde{f}:F\rightarrow C$ extending $f$ given by $\tilde{f}((a_j))=f((a_j))$. The $J$-linearity of $f$ shows that $K\subseteq\ker\tilde{f}$, so that $\tilde{f}$ factors uniquely through $\bar{f}:F/K\rightarrow C$ as required.

		This gives us a natural bijection
		\[J\text{-lin}\left((A_j)_{j\in J};C\right)\leftrightarrow\Ab\left(\bigotimes_{j\in J}A_j,C\right)\]
		showing that $\bigotimes A_j$ represents $J\text{-lin}((A_j);-)$. This justifies that our construction creates a tensor product.

		\subparagraph{}
		Now, let $(R_j)_{j\in J}$ be a family of commutative rings and let $R=\bigotimes_{j\in J}R_j$ be the tensor product of their underlying abelian groups. We will construct a multiplication operation on $R$ that makes $R$ into a ring.

		This multiplication operation will be built from the function
		\[\mu:\prod_{j\in J}R_j\times\prod_{j\in J}R_j\rightarrow R=\bigotimes_{j\in J}R_j\]
		defined by $\mu((a_j),(b_j))=\otimes_j(a_jb_j)$. Notice that if we fix the lefthand side $a=(a_j)_{j\in J}$, then the function $\mu_a:\prod_jR_j\rightarrow R$ given by $\mu_a((b_j))=\otimes_j(a_jb_j)$ is $J$-linear. To check this, let $(b_j),(b_j^\prime),(b_j^{\prime\prime})\in\prod_jR_j$ be such that there is some $i\in J$ such that $b_i=b_i^\prime+b_i^{\prime\prime}$ while $b_j=b_j^\prime=b_j^{\prime\prime}$ for $j\ne i$. Then the $i$-component of $\otimes_j(a_jb_j)$ is $a_ib_i=a_ib_i^\prime+a_ib_i^{\prime\prime}$ so that
		\[\otimes_{j\in J}(a_jb_j)=\otimes_{j\in J}a_jb_j^\prime+\otimes_{j\in J}a_jb_j^{\prime\prime}\]
		as required.

		So, $\mu_a$ factors uniquely through a group homomorphism $\bar{\mu}_a:R\rightarrow R$ for which $\bar{\mu}_a(\otimes_jb_j)=\otimes_j(a_jb_j)$. For arbitrary elements of $R$, this becomes
		\[\bar{\mu}_a:\sum_{k=1}^n\otimes_{j\in J}b_{kj}\mapsto\sum_{k=1}^n\otimes_{j\in J}(a_jb_{kj}).\]

		So, we have a well-defined function
		\[\hat{\mu}:\left(\prod_{j\in J}R_j\right)\times R\rightarrow R\]
		given by
		\[\left((a_j)_{j\in J},\sum_{k=1}^n\otimes_{j\in J}b_{kj}\right)\mapsto\sum_{k=1}^n\otimes_{j\in J}(a_jb_{kj}).\]
		Fixing the right-hand side $b=\sum_{k=1}^n\otimes_jb_{kj}$ we obtain a function
		\[\hat{\mu}_b:\prod_{j\in J}R_j\rightarrow R\]
		given by
		\[(a_j)_{j\in J}\mapsto\sum_{k=1}^n\otimes_{j\in J}(a_jb_{kj}).\]
		As above, this is a $J$-linear function and so factors through a function $\bar{\mu}_b:R\rightarrow R$ that when applied to an arbitrary element of $R$ is given by
		\[\bar{\mu}_b:\sum_{i=1}^m\otimes_ja_{ij}\mapsto\sum_{i=1}^m\sum_{k=1}^n\otimes_j(a_{ij}b_{kj}).\]
		Letting $b\in R$ vary, we obtain a well-defined function $\bar\mu:R\times R\rightarrow R$
		\[\bar{\mu}\left(\sum_{i=1}^m\otimes_ja_{ij},\sum_{k=1}^n\otimes_jb_{kj}\right)=\sum_{i=1}^m\sum_{k=1}^n\otimes_j(a_{ij}b_{kj}).\]
		As in the earlier case of a tensor product of two rings, it is easy to check that $\mu$ is an associative and commutative binary operation that distributes over addition in $R$. Furthermore, $1:=\otimes_j1$ is a multiplicative identity for $\mu$, so that $R$ is a commutative ring.

		\subparagraph{}
		For each $i\in J$, let $\tilde{\iota}_i:R_i\rightarrow R$ be the function defined by $\tilde{\iota}_i(a)=\otimes_ja_j$ where $a_i=a$ and $a_j=1\in R_j$ for $j\ne i$. Then the basic properties of the tensor product show that $\tilde{\iota}_i(a+b)=\tilde{\iota}_i(a)+\tilde{\iota}_i(b)$. The multiplication law described above shows that $\tilde{\iota}_i(ab)=\tilde{\iota}_i(a)\tilde{\iota}_i(b)$. Finally, $\tilde{\iota}_i(1)=1$. So, $\tilde{\iota}_i$ is a ring homomorphism.

		However, $R$ equipped with the homomorphisms $\tilde{\iota}_j$ is not in general the coproduct of the $R_j$! Let $R^\prime\subseteq R$ be the smallest subring of $R$ that contains all of the images of the $\tilde{\iota}_j$. (This should remind you of direct sums of modules \dots .) That is,
		\[R^\prime=\left\{\sum_{i=1}^m\otimes_ja_{ij}\big{|}a_{ij}=1\text{ for all but finitely many }a_{ij}\right\}.\]
		(If $J$ is finite, then $R^\prime=R$.) Then each $\tilde{\iota}_i$ factors uniquely and obviously through a ring homomorphism $\iota_i:R_i\rightarrow R^\prime$ defined in the same way as $\tilde{\iota}_i$. Furthermore, note that for any simple tensor $\otimes_ja_j\in R^\prime$, since $a_j=1$ for all but finitely many $j$, it makes sense to write:
		\[\otimes_ja_j=\prod_{j\in J}\iota_j(a_j).\]
		The product is really a finite product, since only finitely many factors are not the identity element.

		\subparagraph{}
		I claim that $R^\prime$ equipped with the homomorphisms $\iota_i:R_i\rightarrow R^\prime$ is the coproduct of the family of commutative rings $(R)_{j\in J}$.  To see this, let $S$ be another commutative ring and let $\phi_i:R_i\rightarrow S$ be ring homomorphisms for $i\in J$. We must see that there is a unique ring homomorphism $\psi:R^\prime\rightarrow S$ such that $\psi\iota_i=\phi_i$ for each $i\in J$. If there is any such ring homomorphism, then it must satisfy:
		\[\psi\left(\sum_{i=1}^m\otimes_ja_{ij}\right)=\sum_{i=1}^m\psi(\otimes_ja_{ij})=\sum_{i=1}^m\psi\left(\prod_{j\in J}\iota_j(a_{ij})\right)=\sum_{i=1}^m\prod_{j\in J}\psi\iota_j(a_{ij})=\sum_{i=1}^m\prod_{j\in J}\phi_j(a_{ij}).\]
		This gives a formula for the only possible ring homomorphism $\psi:R^\prime\rightarrow S$ satisfying our constraints. It remains to see that $\psi$ is well-defined and really is a ring homomorphism.

		The strategy that we would {\it like} to follow is to show that $\psi$ is induced by a $J$-linear function $\prod_{j\in J}R_j\rightarrow S$ defined by $(a_j)\mapsto\prod_j\phi_j(a_j)$. This will work when $J$ is finite, but this function is not defined on all of $\prod_jR_j$ when $J$ is infinite! So, let us proceed with care.

		\subparagraph{}
		For an element $(a_j)\in\prod_{j\in J}R_j$, say that the \emph{support} of $(a_j)$ is $\{j\in J|a_j\ne 1\}$. Say that $(a_j)$ is of \emph{finite support} if its support is a finite set and of \emph{infinite support} if its support is an infinite set. Then $\prod_{j\in J}R_j$ is a disjoint union of the two subsets $(\prod_jR_j)^f$ of elements of finite support and $(\prod_jR_j)^\infty$ of elements of infinite support.

		Recall that $F$ is the free abelian group with basis $\prod_jR_j$. Splitting the basis into these two subsets, we may write $F=F^f\oplus F^\infty$ where $F^f$ is a free abelian group with basis $(\prod_jR_j)^f$ and $F^\infty$ is a free abelian group with basis $(\prod_jR_j)^\infty$. Recall that $R=F/K$ where $K$ is a subgroup that we will recall in a moment. Notice that $R^\prime$ is the image of $F^f$ modulo $K$. The kernel of the induced group homomorphism $F^f\rightarrow R^\prime$ is $F^f\cap K$, which we will call $K^f$. So, $R^\prime\simeq F^f/K^f$.

		Recall that $K\subseteq F$ is the subgroup generated by all expressions of the form
		\[(a_j)_{j\in J}-(b_j)_{j\in J}-(c_j)_{j\in J}\]
		such that there is some $i\in J$ for which $a_i=b_i+c_i$ and $a_j=b_j=c_j$ for all $j\ne i$. Notice that the supports of $(a_j), (b_j)$ and $(c_j)$ in this expression can differ only in that $i$ might be in some but not others. So, all three of $(a_j),(b_j)$ and $(c_j)$ are in $F^f$ or all three are in $F^\infty$. An arbitrary element of $K^f$ is a sum of finitely many generators above or their inverses with the property that all terms with infinite support cancel out. We may split the sum into a sum of those generators involving terms of finite support and a sum of those generators involving terms of infinite support. The latter sum must be zero. So, we see that $K^f$ is generated by expressions as above involving only terms of finite support.

		To sum up, we have a homomorphism of abelian groups $\otimes:F^f\rightarrow R^\prime$ given by $(a_j)_{j\in J}\mapsto\otimes_{j\in J}a_j$. The kernel $\otimes$ is the subgroup $K^f$ generated by expressions of the form
		\[(a_j)_{j\in J}-(b_j)_{j\in J}-(c_j)_{j\in J}\]
		such that there is some $i\in J$ for which $a_i=b_i+c_i$ and $a_j=b_j=c_j$ for all $j\ne i$ and $(a_j), (b_j)$ and $(c_j)$ all have finite support.

		Now, we do have a function $\tilde{\psi}:(\prod_jR_j)^f\rightarrow S$ given by $\tilde{\psi}((a_j))=\prod_{j\in J}\phi_j(a_j)$. This function is well-defined because each $(a_j)\in(\prod_jR_j)^f$ has finite support and $\phi_j(1)=1$. Since $F^f$ is a free abelian group with basis $(\prod_jR_j)^f$, $\tilde{\psi}$ induces a group homomorphism $\hat{\psi}:F^f\rightarrow S$ given by
		\[\hat{\psi}\left(\sum_{i=1}^m\alpha_i(a_{ij})_{j\in J}\right)=\sum_{i=1}^m\alpha_i\tilde{\psi}((a_{ij})_{j\in J})=\sum_{i=1}^m\alpha_i\prod_{j\in J}\phi_j(a_{ij}).\]
		where $\alpha_i\in\ZZ$. To see that $\tilde{\psi}$ factors through $F^f/K^f\simeq R^\prime$, we must check that each generator of $K^f$ maps to $0$. This follows from $\phi_i$ being a homomorphism on abelian groups and the distributive property of multiplication over addition in $S$.

		This is enough to see that $\psi:R^\prime\rightarrow S$ as given above is well-defined and is a homomorphism of abelian groups under addition. Now, we must see that it is also a homomorphism or monoids under multiplication. First note that $\psi(1)=\prod_{j\in J}\phi_j(1)=1$ so that $\psi$ preserves the identity. Now, we check directly that $\psi$ preserves multiplication:
		\[\psi\left(\sum_{i=1}^m\otimes_ja_{ij}\right)\psi\left(\sum_{k=1}^n\otimes_jb_{kj}\right)
		=\left(\sum_{i=1}^m\prod_{j\in J}\phi_j(a_{ij})\right)\left(\sum_{k=1}^n\prod_{j\in J}\phi_j(b_{kj})\right)\]
		\[=\sum_{i=1}^m\sum_{k=1}^n\prod_{j\in J}\phi_j(a_{ij})\phi_j(b_{kj})\\
		=\sum_{i=1}^m\sum_{k=1}^n\prod_{j\in J}\phi_j(a_{ij}b_{kj})=\psi\left(\sum_{i=1}^m\sum_{k=1}^n\otimes_j(a_{ij}b_{kj})\right).\]

		Now, we have seen that for any family of ring homomorphisms $\phi_j:R_j\rightarrow S$ there is a unique ring homomorphism $\psi:R^\prime\rightarrow S$ such that $\phi_j=\psi\iota_j$ for every $j\in J$. Thus, $R^\prime$ is the coproduct of the $R_j$ and we may write $R^\prime=\coprod_{j\in J}R_j\subseteq\bigotimes_{j\in J}R_j$ with a clean conscience.
		\end{document}
