\documentclass[main.tex]{subfiles}

\begin{document}

\paragraph{}
\begin{exercise}
	Consider a commutative rectangle
	\[\begin{tikzcd}[cramped]
			\bullet \ar[r] \ar[d] &
			\bullet \ar[r] \ar[d] \ar[dr, phantom, "\lrcorner", very near start] &
			\bullet \ar[d] \\[0.6em]
			\bullet \ar[r] &
			\bullet \ar[r] &
			\bullet
	\end{tikzcd}\]
	whose right-hand square is a pullback. Show that the left-hand square is a
	pullback if and only if the composite rectangle is a pullback.
\end{exercise}

\begin{proof}
	For ease of reference we will give the objects in the above diagram names as
	follows.
	\[\begin{tikzcd}[cramped]
			a \ar[r] \ar[d] &
			b \ar[r] \ar[d] \ar[dr, phantom, "\lrcorner", very near start] &
			w \ar[d] \\[0.6em]
			x \ar[r] &
			y \ar[r] &
			z
	\end{tikzcd}\]
	Recall that a pullback is the limit of a functor from
	\(\bullet\to\bullet\gets\bullet\), so we are dealing with three functors
	from this category with the following images:
	\[x\to z\gets y,\quad y\to z\gets w,\qtextq{and} x\to y\gets b .\] Note that
	the map from \(x\) to \(z\) factors through \(y\).

	Further note that because \(b\) is a pullback and \(a\) is the apex of cone
	with \(y\to z\gets w\) as a base (with the composite maps \(a\to x\to y\)
	and \(a\to b\to w\) as legs), the map \(a\to b\) is unique.

	\[\begin{tikzcd}[cramped]
			c \ar[dr, dashed] \ar[drr, dashed, bend left=15]
			\ar[drrr, bend left=20] \ar[ddr, bend right=15] \\ &
			a \ar[r] \ar[d] &
			b \ar[r] \ar[d] &
			w \ar[d] \\[0.5em] &
			x \ar[r] &
			y \ar[r] &
			z
	\end{tikzcd}\]
	Suppose first that the left-hand square is a pullback along with the
	right-hand square, and there is an apex \(c\) to the cone with
	\(x\to z\gets w\) as a base.
	By taking \(c\to x\to y\) as a leg, \(c\) is a cone over
	\(y\to z\gets w\) and thus there is a unique map \(c\dashrightarrow b\)
	making \(c\) also the apex of a cone over \(x\to y\gets b\). Thus there is a
	unique map \(c\dashrightarrow a\) making the whole diagram commute. Thus
	\(a\) is the limit of \(x\to z\gets w\).
	
	Conversely suppose that the whole diagram is a pullback and that \(c\) is a
	cone over \(x\to y\gets b\). Then \(c\) is also a cone over
	\(x\to z\gets w\) with \(a\to b\to w\) as a leg. Thus we again have a unique
	map \(c\dashrightarrow a\) that makes the diagram commute, and \(a\) is also
	the limit of \(x\to y\gets b\).
\end{proof}
\end{document}
