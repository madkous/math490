\documentclass[../../main]{subfiles}
\begin{document}

\begin{exercise}
	Suppose $\func{E}{\mathsf{I}}{\sJ}$ defines an equivalence between
	small categories and consider a diagram $\func{F}{\mathsf{J}}{\sC}$.
	Show that the category of $\sJ$-shaped cones equivalent to the category
	of $\mathsf{I}$-shaped cones over $FE$, and use this equivalence to
	describe the relationship between limits of $F$ and limits of $FE$.
\end{exercise}

\begin{proof}
	We will denote the category of cones over some diagram $\func{G}{\sJ}{\sC}$
	as $\mathsf{Cone_{G}}$  Recall the following facts for $\mathsf{Cone_{G}}$ :
	\begin{enumerate}
		\item Objects of $\mathsf{Cone_{G}}$ are natural transformations
			$\Func{\al}{c_{\sJ} }{G}$ where $c_{\sJ}$ is the constant functor
			mapping every object in $J$ to $c$ and every morphism to $1_{c}$.

		\item A morphism $\func{f}{\al}{\be}$ between cones $\Func{\al}{c_{\sJ}
			}{G}$ and $\Func{\be}{d_{\sJ} }{G}$ is the natural transformation
			$\Func{ f_{\sJ} }{d_{\sJ}}{c_{\sJ}}$ where every component of $f_{\sJ}$ is
			$f$ and $\al \cdot f_{\sJ} = \be$.

		\item The identity morphism of $\al$ is simply $1_{c_J}$ where $c_{J}$ is
			the domain of $\al$.
		\item Composition of morphisms is composition of natural transformations.
	\end{enumerate}

	Now we will define the functor
	$\func{RW_{E}}{\mathsf{Cone_{F}}}{\mathsf{Cone_{FE}}}$ as follows:
	\begin{enumerate}
		\item $RW_{E}$ maps cone $\al$ to $\al E$ where for $i \in
			\ob\mathsf{I} $ the corresponding component is
			$\func{\al_{Ei}}{c}{FEi}$, in other words, $\al E$ is a restriction
			of $\al$ to the image of the functor $E$.
		\item $RW_{E}$ maps morphism $f_{J}$ to $f_{I}$.

		\item $RW_{E}$ performs a right whiskering of $E$ onto natural
			transformations.
	\end{enumerate}

	$RW_{E}$ is clearly a functor since the composition $g_{J} \cdot f_{J}$
	gets
	mapped to $g_{I} \cdot f_{I}$ and $1_{c_{J}}$ gets mapped onto
	$1_{c_{I}}$. Now we will show that $RW_{E}$ is an equivalence. First we
	will show essential surjectivity. Suppose $\gm$ is an object in
	$\mathsf{Cone}_{FE}$, we will construct a  natural transformation
	$\gm^{F}$ such that $ \gm^{F}_{j} = F\phi_{j} \cdot \gm_{i}$ where $i$
	is an object in $\mathsf{I}$ such that $Ei \cong j$. and
	$\func{\phi_{j}}{Ei}{j}$ is an isomorphism. When $Ei = j$, we will let
	$\phi_{j} = 1_{j}$. Our construction is well-defined since $E$ is
	essentially surjective. To show that $\gm^{F}$ is a natural
	transformation, take a morphism $\func{a}{j}{j'}$ in $\sJ$. We can show
	that $a = \phi_{j'}b\inv\phi_{j}$ for a unique $b$ in $\sJ\qty{Ei,Ei'},$
	where $Ei \cong j$ and $Ei' \cong j'.$ We can form a bijection from
	$\sJ\qty{Ei,Ei'}$ to $\sJ\qty{j,j'}$ by pre-composing $\inv\phi_{j}$ and
	post-composing $\phi_{j'}$. Furthermore since $E$ is full and faithful,
	there exist a
	unique $c \in \mathsf{I}\qty{i,i'}$ such that $b = Ec$. We can
	claim that $a = \phi_{j'}Ec\inv\phi_{j}$ for a unique $c \in
	\mathsf{I}\qty{i,i'}$. Now we must show that $Fa \cdot \gm^{F}_{j} =
	\gm_{j'}^{F}$, but this is equivalent to $FEc \cdot \gm_{i} = \gm_{i'}$
	which follows from naturality of $\gm$. Thus $\gm^{F}$ is natural and is
	a cone over $F$. Applying $RW_{E}$ to $\gm^{F}$ gives us $\gm$ since
	each $\phi_{Ei}$ for $i \in \ob\mathsf{I}$ is the identity. Thus
	$RW_{E}$ is essentially surjective.

	To show full and faithful take cones $\al$ and $\be$ and suppose $f_{J}$
	and $g_{J}$ are morphisms from $\al$ to $\be$ such that $RW_{E}f_{J} =
	RW_{E}g_{J}$. Then $f_{I} = g_{I}$ which means that $f = g$. Thus $f_{J}
	= g_{J}$. Now take a morphism $h_{I}$ from $\al E$ to $\be E$, we must
	show that $h_{J}$ is a morphism from $\al$ to $\be$. Since we already
	have that $\al_{Ei} \cdot h = \be_{Ei}$ for all $i \in \ob\mathsf{I}$,
	by naturality, for each $\func{k}{Ei}{j}$ in $\sJ$, we have that $Fk
	\cdot \al_{Ei} = \al_{j}$ and $Fk \cdot \be_{Ei} = \be_{j}$. Since, $E$
	is essentially surjective, for any $j \in \ob\sJ$, we have an
	isomorphism $\func{\phi_{j}}{Ei}{j}$ for some $i$ allowing the equality
	$F\phi_{j} \cdot \al_{Ei} \cdot h = F\phi_{j} \cdot \be_{Ei}$ which
	gives us $\al_{j} \cdot h= \be_{j}$ for all $j$. Thus $h_{J}$ is a
	morphism from $\al$ to $\be$.

	Thus $RW_{E}$ is full, faithful, and essentially surjective. Therefore,
	$\mathsf{Cone}_{F}$ is equivalent to  $\mathsf{Cone}_{FE}$.

	Suppose $F$ had a limit cone $\om_{F}$, then by the universal property
	of limits, $\om_{F}$ is terminal in $\mathsf{Cone}_{FE}$. Since $RW_{E}$
	is full, faithful, and essentially surjective then for all cones $\be$
	over $FE$, there exist a cone $\al$ of $F$ such that
	$\mathsf{Cone}_{F}\qty{\al,\om_{F}} \cong \mathsf{Cone}_{FE}\qty{\al E,
	\om_{F} E}  \cong \mathsf{Cone}_{FE}\qty{\be,\om_{F}E}$. Then
	$\om_{F}E$ is the terminal object of $\mathsf{Cone}_{FE}$. Thus $\om_{F}E$
	is the limit cone of $FE$. The components of $\om_{F}$ and $\om_{F}E$ have
	the same domain since $\om_{F}E$ is a restriction of the components of
	$\om_{F}$. Therefore $F$ and $FE$ have the same limit.

\end{proof}

\end{document}
