\documentclass[../../main]{subfiles}

\begin{document}

\paragraph{}
\begin{exercise}
	Let $ G $ be regarded as a 1-object category $ \mathsf{B}G $. Describe the
	colimit of a diagram $ \mathsf{B}G \to \mathsf{Set} $ in group-theoretic
	terms, as was done for the limit in Example 3.2.12.
\end{exercise}
\paragraph{Proof.}

For a diagram $ D: \mathsf{B}G \to \mathsf{Set} $, note that
\[\mathrm{colim}\ D = {\frac{ \underset{k \in \mathsf{B}G}{\coprod} Dk}{\sim}} \]
where $ \sim $ is the smallest equivalence relation on the coproduct such that
for every $ g: i \to j \in \mathsf{B}G $ and $ n \in Di $, $ \iota_{Dj} Dg(n)
\sim \iota_{Di} n $. So, for example, the following diagram illustrates a part
of the diagram $ D $:
\[\begin{tikzcd}
		Di \arrow[rr, "g"] \arrow[rd, "\iota_{Di}"] &  & Dj \arrow[ld, "\iota_{Dj}"'] \\ &
		\underset{k \in \mathsf{B}G}{\coprod} Dk &
\end{tikzcd}\]

But in this case, the only object in our diagram is a set $ X $, and the
morphisms in our diagram are simply the left actions of elements of $ G $
applied to a set $ X. $ The following specializes the above diagram to this
specific case:

\[\begin{tikzcd}
	X \arrow[rr, "g"] \arrow[rd, "\iota_{X}"] &  &
	X \arrow[ld, "\iota_{X}"'] \\ & X &
\end{tikzcd}\]

Of course, the inclusion map $ \iota_X: X \to X $ is simply the identity, so we
need to have that $ g(x) \sim x $ in colim $ D $. This means each equivalence
class must consist of an element $ x \in X $, along with $ g(x) $ for every $ g
\in G $. In other words, $ [x] = \{x, g(x) | g \in G\}, $ also known as the
orbit of $ x $. So the set of all equivalence classes (that is, the colimit we
are looking for) is simply the orbit space (that is, the set of all orbits) of
the action of $ G $ on $ X $.

\end{document}
