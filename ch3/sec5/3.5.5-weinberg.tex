\documentclass[../../main]{subfiles}
\author{Teddy Weinberg}
\begin{document}
\paragraph{}
\begin{exercise}
Describe the limits and colimits in the poset of natural numbers with the order
relation $k \le n$ if and only if $k$ divides $n$.
\end{exercise}

\begin{proof}
Recall that when considering a poset as a category, a morphism $f\colon c
\rightarrow d$ exists if and only if $c \le d$, and that $f$ is unique for the
given domain and codomain.  In this case, $c \le d$ means $c$ divides $d$.  The
limit of the diagram must be able to divide everything in the diagram.  Because
of the structure of the morphisms, commutativity and uniqueness are not issues.
We need to find an element that divides everything in the diagram, but also is
universal in the sense that anything else that divides everything in the
diagram also divides our limit.  Thus, we choose the greatest common divisor of
the objects in the diagram.  The greatest common divisor is divisible by any
smaller divisor, and thus will be our limit. 


Similarly, for colimits we need to find an object that is divided by every
object in the diagram.  It also must divide any natural number that is divided
by everything in the diagram.  Thus, we must choose our colimit as the least
common multiple, which divides any other multiple.  

Now we consider the empty diagram as a special case, as the concepts of GCD and
LCM would not apply.  In the case of the empty diagram, the limit would be $0$,
which can be divided by every other natural number, and thus is a terminal
object. The colimit would be $1$, as it divides everything, and is our initial
object.
\end{proof}

\end{document}
