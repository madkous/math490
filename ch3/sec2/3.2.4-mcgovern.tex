\documentclass[../../main]{subfiles}
\begin{document}

\paragraph{}

\begin{exercise}
	Generalize Exercise 3.2.iii to show that for any small category $\sJ$, any
	locally small category $\sC$ and any paralell pair of functors $F,G\colon
	\sJ \rightarrow \sC$, the set $\Hom(F,G)$ of natural transformations can be
	defined as a small limit in $\Set$. (Hint: the diagram whose limit is
	$\Hom(F,G)$ is indexed by a category $\sJ^{\S}$ whose objects are morphisms
	in $\sJ$ and which had morphisms $1_x \rightarrow f$, $1_y \rightarrow f$
	for every $f\colon x\rightarrow y$ in $\sJ$)
\end{exercise}

\begin{proof}
	First, we note that the underlying shape of the diagram in $\Set$ will be
	based on the following diagram in $\sJ^{\S}$:
	\begin{align}
		\xymatrix{
		1_{x} \ar[r] & f & \ar[l] 1_y}
	\end{align}
	Now, we define a functor $H\colon \sJ^{\S} \rightarrow \sC$ as the following:
	\begin{itemize}
		\item For $f \in ob\sJ^{\S}$ where $f\colon x \rightarrow y$, $Hf = \sC(Fx,Gy)$
		\item For a morphism $\tau\colon 1_x \rightarrow f$ in $J{\S}$ where $f\colon
			\rightarrow y$, $H\tau\colon \sC(Fx,Gx) \rightarrow C(Fx, Gy)$ is defined
			as  post-compostion by $Gf$ and if $\tau: 1_y \rightarrow f$,
			$H\tau\colon \sC(Fy,Gy) \rightarrow \sC(Fx,Gy)$ is defined as precomposition
			by $Gf$.
	\end{itemize}
	We see that the image of $(1)$ under this functor is :
	\begin{align}
		\xymatrix{
			\sC(Fx,Gx) \ar[r]^{Gf \circ -} & \sC(Fx, Gy)  &\ar[l]^{- \circ Ff} \sC(Fy, Gy)
		}
	\end{align}

	Now, we must show that $\Hom(F,G)$ is a cone over this diagram. To do this, we
	construct a family of functions $\phi_{xy}\colon \Hom(F,G) \rightarrow \sC(Fx,Gy)$
	where $\phi_{xy}(\eta) = \eta_x$, the corresponding component of the natural
	transformation (if this is the case, we denote it as $\phi_{x}$ if $x = y$ and
	$\phi_{xy}(\eta) = Gf\eta_x$ if $x \neq y$. Because we have that $\eta$ is a
	natural transformation and that $Gf\eta_x = \eta_yFf$, we see that we have
	commutativity of:

	\begin{align}
		\xymatrix
		{ & \Hom(F,G) \ar[dl]_{\phi_x} \ar[dr]^{\phi_y} \ar[d]_{\phi_{xy}}  & \\
			\sC(Fx,Gx) \ar[r]^{Gf \circ -} & \sC(Fx, Gy)  &\ar[l]^{- \circ Ff} \sC(Fy, Gy)
		}
	\end{align}

	Now, we must show that $\Hom(F,G)$ is the limit for this diagram. Suppose that
	there is an object $X \in \Set$ that also forms a cone over our diagram. This
	means that for every object $f \in \sJ^{\S}$ and $Hf = \sC(Fx, Gy)$, we have a
	$\sigma_{x,y}\colon X \rightarrow \sC(Fx,Gy)$ so that $\sigma_{xy}(k)\colon Fx \rightarrow
	Gy$ and in particular, if $x=y$, then $\sigma_{x}(k)\colon Fx \rightarrow Gx$. Now,
	suppose we have a $r: X \rightarrow \Hom(F,G)$, such that $\sigma_{x} =
	\phi_{x}r$. So for $k \in X$ $\phi_{x}r(k)  = \sigma_{x}(k)$. Now, consider
	that $\phi_x$ is selecting the $x$ component of the natural transformation
	$r(k)$, so $(r(k))_x = \sigma_{x}(k)$. So, we see what the construction of $r$
	must be. This is the only possible construction that satisfies $\phi_{x}r =
	\sigma_x$. We can also easily see that $\phi_{xy}r = \sigma_{xy}$ with our
	construction of $r$. So we have shown that any cone with apex $X$ factors
	uniquely through $\Hom(F,G)$ and so $\Hom(F,G)$ is the limit of this diagram.

\end{proof}
\end{document}
