\documentclass[../../main]{subfiles}
\begin{document}

\paragraph{}
\begin{exercise}
	 Show that for any small diagram $ F\colon \sJ \to \Set $, the equalizer 
	 diagram (3.2.14) can be modified to yield a slightly smaller equalizer 
	 diagram:
	$$ \textbf{Matt will add this diagram using TikZ}$$
	in which these condproduct is indexed only by non-identity morphisms. 
\end{exercise}

\begin{proof}
	To show that we can slightly shrink the equalizer diagram, we will review
	the proof of Theorem 3.2.6 and notice that removing the objects from the 
	second product does not lose any information. Recall that the objects of 
	a category can be identified with their identity morphisms. Notice that
	the cone over $ F $ is indexed by the non-identity elements. When $ f $
	is an identity morphism the diagram
	$$
	\xymatrix{&1\ar[dr]^{\lm_{\dom f}}\ar[dl]_{\lm_{\cod f}}\\
		F(\dom f)\ar[rr]_{Ff}&&F(\cod f)
	}
	$$
	collapses into a single line. Thus when defining the parallel pair of 
	morphisms of $ c\AND d $ if $ f $ is an identity morphism (an object of $ 
	\sJ $) the legs of the cone are the same, so we do not lose any information excluding them.   
\end{proof}

\end{document}
